\chapter{Introduction}

Large software systems are a composition of intertwined units of functionality,
often related in complex ways. They can be developed across many years and by 
large teams. It is no surprise that complexity arises quite easily. 

It becomes progressively harder to add new features and improvements while being 
sure that everything else works as it should. That is, the software remains
correct. To tackle this problem of complexity, many strategies and different 
angles have been studied and implemented by large-scale software systems. 

Automated regression testing is one of the main strategies widely adopted by the industry,
to ensure that changes to the code will not negatively affect the users of the system. 

It consists of developers writing automated tests: code that executes scenarios that a user would experience
and asserts that desired properties hold. Afterwards, when changes to the code are done,
executing successfully the set of automated tests gives confidence that such scenarios will
not be impacted by the change.

In practice, this is an effective technique to keep the quality of the system and
allow for its evolution. However, there is a drawback, as the system evolves, more
scenarios get added and eventually, it can become time-consuming to run the complete
set of automated tests.

Another challenge in software engineering consists in having an effective development process when
working with large teams. Many strategies have also been studied, and the industry 
consensus is to use \emph{Continuous Integration} (CI) processes.

The main idea of the CI software development process is that developers should
integrate code changes frequently. When developers integrate their changes,
they make other developers on the team aware of their changes. The other developers
can then react accordingly if the change conflicts with their work.

Frequently integrating changes to the product increases the risk of making mistakes.
For that reason, as a safeguard, the set of automated tests should be run before 
integrating every change. The process of building and testing the code automatically
every time a change is meant to be integrated is called a CI pipeline.

By frequently integrating changes with the regression test suite backing its correctness
we ensure quality in every small step.

We can now see that having a time-consuming set of tests to execute is a relevant 
drawback for teams that want to adopt CI practices. While on one hand, it is desired
to automatically run tests with every small change, a sufficiently large test suite
can take a considerable amount of time to execute.

To address this problem, academic research and industry have focused on several angles.
Algorithms on Test Selection and Prioritization (TSP) are techniques that aim to 
automatically propose a relevant sorted subset of tests to execute. These techniques
acknowledge that it is not needed to run all the tests, instead, only
those which would be more likely to fail first. As the goal of running a test suite is to
avoid potential errors to be integrated into the codebase, we can give the developer 
feedback on errors to be corrected earlier, if found earlier.

For this project, TSP techniques are proposed and evaluated for one of the CI pipelines
of the \emph{Business Central} project. \emph{Business Central} is an Enterprise Resource Planning (ERP)
software product by Microsoft. The aim is to find an adequate technique
for the product and to propose possible directions the product could take to improve the
CI cycle feedback time and effectiveness.

Not only is optimizing CI pipelines relevant to improving the agility of software development, 
but it also has a positive impact on the energy resources they require.
These pipelines are run multiple times a day, and the cost of operating the infrastructure
required to keep the system evolving is significant.

In this project, we will follow the same approach as previous research by Bertolino
et.al. in \cite{Bertolino2020LearningtoRankVR}, transforming this problem into a \emph{ranking} problem.
Afterwards, different \emph{Learning to Rank} algorithms will be applied and evaluated.
However, in contrast to their work, we use test coverage information for two different 
parameters of how our ranking datasets are created.

We also outline the different challenges and technical difficulties that adapting
academic techniques posed when taking into consideration the practical realities
of an industrial environment. 

All the code for the different stages of the project can be found 
in the accompanying repository.\footnote{https://github.com/mynjj/msc-thesis-public}