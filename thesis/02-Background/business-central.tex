\section{Business Central}

\emph{Business Central}(BC) is an Enterprise Resource Planning (ERP) software 
system targeting Small and Midsized Businesses from Microsoft. It's functionality
spans several areas of a company like Finance, Sales, Warehouse Management, among
many others.

BC allows for Microsoft partners to provide custom extensions that suit customer
needs as much as required. This is done through application extensions that developers
can write in AL: a Domain Specific Language (DSL) for the application logic. 
These extensions modify the experience end-customers have with the product, and 
enhance the product with any custom logic their business may require.
Furthermore, not only partners provide extensions for BC, all the business logic
are first-party extensions that Microsoft developers maintain for the core business 
functionality of the product.

In this thesis project we will focus on the CI pipeline and regression tests
used for the business logic code (also referred to as \emph{application} code).
The project has several other pipelines for the different parts of the product,
but for the scope of the project we focus on optimizing the CI pipeline through
techniques for test selection and prioritization targeting changes in the AL DSL.

\subsection{The \emph{application} CI pipeline and the DME system}\label{s:bc-ci-dme}

The BC \emph{application} code follows a traditional CI cycle for development, which we describe 
for completeness and clarity. Whenever developers complete tasks by making new 
changes to the \emph{application} code, they create a Pull Request on the 
Version Control System (VCS) for the desired target branch. Before this request
succeeds and the code is merged, certain checks have to be fulfilled, which 
include the execution of automated tests.

To execute automated tests, a \emph{job} is started on the Distributed Model Executor (DME)
system, the internal build system for the product.  A job is defined by a set of 
tasks, defined by scripts that execute the different required stages. Such specification is
called the \emph{model} of the job. This set of tasks may have dependencies between each other. 
We can represent this set of tasks with dependencies as a Directed Acyclic Graph (DAG), where
each task correspond to a node and an edge corresponds to the target depending on the source.

All these tasks, and the job definition are defined within the same
Version Control repository as code, giving to the application developers a complete 
control over job's execution. A Pull Request triggering the CI pipeline is not the only way 
a developer can request \emph{jobs}. They can also require them whenever on the development 
cycle for their convenience.

Some of these tasks, correspond to running different groups of tests. Therefore, the cost of executing 
certain tests belonging to a task, depends on the dependencies the task has.
We explain by giving an example: Suppose a \emph{job} being executed 
is defined by a model consisting of the tasks shown in figure \ref{fig:example-dag-tasks} as the 
underlying DAG for such set of task-dependencies. 

\begin{figure}
    \def\svgwidth{\columnwidth}
    \includesvg[inkscapelatex=false]{thesis/figures/network-plots/example-tasks}
    \caption{Example of tasks that depend on each other for a job execution represented as a DAG.}
    \label{fig:example-dag-tasks}
\end{figure}

We can see that the task \texttt{RunApplicationTests} depends on both \texttt{BuildTests}
and \texttt{BuildApplicationTestsDatabase}. On the other hand, the task 
\texttt{RunSystemTests} only depends on \texttt{BuildTests}.

In this example, the DME system could assign two different machines to execute 
\texttt{BuildTests} and \texttt{BuildApplicationTestsDatabase}. Lets suppose that
\texttt{BuildApplicationTestsDatabase} takes longer, when the task \texttt{BuildTests}
is done the task \texttt{RunSystemTests} can already begin as it's only dependency
has finished. However \texttt{RunApplicationTests} will have to wait for it's other
dependency to start execution. In this scenario, running tests in the task \texttt{RunApplicationTests}
has a higher cost than running tests in \texttt{RunSystemTests}.

This means that when executing test selection or prioritization considering the complete
set of tests. Some of them entail a higher computing cost depending on the dependencies they
have.

The DAG of task-dependencies for the \emph{Application} pipeline of \emph{Business Central} project is not as simple 
as the given example. In fact, it is so large that we can not show it meaningfully on a
page. However, to satisfy the reader's curiosity, figure \ref{fig:full-job-metamodel-dag} 
shows a complete diagram for the metamodel underlying such execution job.\footnote{A 
metamodel in this context is a specification for the model that the job actually executes, so it is \textit{smaller}, and easier to show.}

\begin{figure}
    \centering
    \def\svgscale{0.1}
    \includesvg{thesis/figures/network-plots/metamodel-nonames}
    \caption{Underlying DAG of the metamodel of job executions of the \emph{Application} pipeline for the \emph{Business Central} product.}
    \label{fig:full-job-metamodel-dag}
\end{figure}

\subsection{Test Selection currently on BC \emph{application} tests}
\label{sec:bg-bc-test-selection-currently}

In the current CI pipeline of the \emph{application} code of the product there is 
a selection method based on test coverage information of each test execution. We will
define more thoroughly the definition of a test selection method in section \ref{s:tsp-tech},
for now it suffices to think of it as selecting just a subset of tests from the complete
set of tests in some meaningful way. In this section, we review how are tests defined for \emph{application} code, and what kind
of coverage information we have available when executing tests.

\subsubsection{Application Tests in AL}\label{sec:app-tests-al}
We first give an overview of how tests are defined in AL.  For a more thorough overview of the AL
language and its' role on the \emph{Business Central} system, we refer to appendix \ref{sec:appendix-allang}.

AL organizes the code in \emph{Objects} \footnote{Note that objects in this context do not 
correspond to objects in the traditional sense of Object Oriented Programming.}, these
\emph{Objects} represent different units of functionality for the different features of the
product, for example tables in a database, or pages the user can interact with. A common 
type of \emph{object} is a \emph{codeunit} which is comprised of different procedures that
can be called from any other \emph{AL object}\footnote{\emph{Modules} is a similar concept analogous
to \emph{codeunits} in other languages like Python, NodeJS, Rust, Haskell, among others.}.

An \emph{application} test is written also in AL, as a \emph{codeunit} with specific annotations for it to 
be identified by test runner applications. In figure \ref{fig:bg-bc-test-codeunit} you can see
an excerpt of a group of tests procedures defined under a test codeunit. In a more general perspective, 
we can think of test codeunits as groups of test scenarios.

\begin{figure}
    \begin{Verbatim}[fontsize=\small]
codeunit 135203 "CF Frcst. Azure AI"
{
    Subtype = Test;
    TestPermissions = NonRestrictive;
    // ...
    [Test]
    procedure AzureAINotEnabledError()
    // ...
    [Test]
    procedure NotEnoughHistoricalData()
    // ...
    [Test]
    procedure FillAzureAIDeltaBiggerThanVariancePercNotInserted()
    // ...
}
    \end{Verbatim}
    \caption{Excerpt of the test codeunit 135203.}
    \label{fig:bg-bc-test-codeunit}
\end{figure}

Test runners are also implemented in AL. These are programs that run the test codeunits, with different
settings, like permission sets to use, or whether or not should the test runner persist changes after 
test execution, which can be desirable according to the scenarios the developer wishes to test. Currently, 
the project has two different implementations of test runners, used for different
sets of tests, result of historical legacy. As means of identification, we identify them 
as \emph{CAL Test Runner} and \emph{AL Test Runner}.

In the current \emph{application} CI pipeline, an \emph{Application Test} task on the job model can use either of these 
two test runners depending on their needs, or historical legacy. In the current model definition,
roughly half of the test tasks are using \emph{CAL Test Runner} and the other half the \emph{AL Test Runner}.

Aside from historic differences these test runners have, for our purposes we highlight a significant
difference: The \emph{CAL Test Runner} can produce coverage information of tests execution, and the \emph{AL Test Runner}
currently in production does not.

\subsubsection{Coverage information}\label{s:bg-bc-coverage}
AL has implemented a primitive of the language that can be used to record information about which lines 
of code of which objects were executed. In a broad sense, this is how the test coverage information from 
the \emph{CAL Test Runner} is produced. After setting up the context for each test, this primitive is used
to record activity on each AL object throughout execution.

The \emph{CAL Test Runner} currently has 3 different kind of outputs relating to coverage information.
We will use information that will allow us to: given a test codeunit executed by the test runner, 
to have a list of the different \emph{AL objects} and lines within these objects that were ran by
the test execution.

For concreteness and further illustrate the information available we show an example. In figure \ref{fig:bg-bc-test-codeunit}
we have an excerpt of the test codeunit 135203, containing several test procedures. In figure \ref{fig:bg-bc-coveragefile} we can see
a corresponding excerpt of its coverage file. Such coverage was collected for a given state of the codebase.

Each row on the coverage file corresponds to a line being executed when executing test codeunit 135203. The first 3 columns 
identify such line by giving the Object Type and Object Id the line belongs to, and the Line Number to be considered. The last
column corresponds to the number of times the line was executed. 

We can see in this example that Codeunit 28 had among others lines 111, 112, 115 and 116  executed. In 
figure \ref{fig:bg-bc-covered-file} we can see such lines.

\begin{figure}
    \begin{Verbatim}[fontsize=\small]
"Table","4","17","1"
"Table","4","18","1"
...
"Table","14","112","1"
"Table","14","127","1"
...
"Codeunit","28","111","1"
"Codeunit","28","112","1"
"Codeunit","28","115","1"
"Codeunit","28","116","1"
"Codeunit","28","119","1"
"Codeunit","28","122","1"
...
    \end{Verbatim}
    \caption{Excerpt of coverage file for the test codeunit 135203. The columns correspond to: Object Type, Object Id, Line Number and Number of Hits}
    \label{fig:bg-bc-coveragefile}
\end{figure}

\begin{figure}
    \begin{Verbatim}[fontsize=\small]
1  codeunit 28 "Error Message Management"
...
106    local procedure GetContextRecID(ContextVariant: Variant; ...
107    var
108       RecRef: RecordRef;
109        TableNo: Integer;
110    begin
111        Clear(ContextRecID);
112        case true of
113            ContextVariant.IsRecord:
114                begin
115                    RecRef.GetTable(ContextVariant);
116                    ContextRecID := RecRef.RecordId;
117                end;
    \end{Verbatim}
    \caption{Some of the lines covered by test codeunit 135203.}
    \label{fig:bg-bc-covered-file}
\end{figure}

\subsubsection{Test Selection with coverage information}
Currently the CI pipeline performs test selection for tasks executed with the \emph{CAL Test Runner},
for which coverage information of each executed test codeunit is recorded.

The selection technique is based in obtaining the lines and files which the developer is
affecting when integrating it's change. And finding such lines on the recorded coverage information.

For all those matching lines, the corresponding test codeunit is selected for execution. This 
effectively selects every test which traversed during its execution the lines being changed
by the developer.

However, for tests being run by the \emph{AL Test Runner}, this selection is not available as the corresponding 
coverage information is not available. We could classify the current technique as a partial selection
based on coverage traces.

As we will explain in section \ref{s:method-collecting-coverage}, we will use this coverage 
information, and add features to the \emph{AL Test Runner} to allow for collecting coverage
traces as well.