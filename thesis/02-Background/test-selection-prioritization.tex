\section{Test Selection and Prioritization techniques}\label{s:tsp-tech}

Several Test Selection and Prioritization techniques have been studied in
the literature both by academia and industry.

In this context, the problem of test selection is defined as: given a change
to the codebase, and a complete set of tests; to obtain a subset of tests,
such that the capacity of fault detection of the test suite is not lost. This 
means that if for such change the test suite execution results in failure, this subset
of tests should fail as well. A stronger condition is given by a \emph{safe} 
selection algorithm \cite{366926}, which requires every failing test case to be included in the selection.

The problem of test prioritization has the same inputs, and it aims to find a
sorting for the tests to be executed that prioritizes running the tests that
are more likely to fail first.

Intuitively our aim in this problem is that: given a change in the codebase,
and a complete test suite to execute; determine a subset (selection) that
does not miss any fault-revealing test case, and a sorting (prioritization)
that increases the probability of finding failures earlier.

It is worth emphasizing that given a prioritization technique, we can induce
selection techniques by selecting the subset of tests that were prioritized the highest. 
The size of such selection could be determined by a given size, duration, or other criteria.
This is how we obtain the selected subsets evaluated in chapter \ref{s:results}.

\subsection{Related work}
\label{sec:bg-tsp-related-work}

Yoo and Harman in their survey \cite{Yoo2012RegressionTM} present a detailed overview of techniques used for
regression test selection, prioritization, and minimization\footnote{Minimization is a problem not dealt with in this project, it consists of removing superfluous tests from a test suite}.
In this study, several foundational works on this area are presented, including formal definitions, and metrics to evaluate this problem.

In the literature, we can find similar techniques to the already implemented selection technique
in the pipeline under study. As explained in section \ref{sec:bg-bc-test-selection-currently},
the CI pipeline of interest of the product collects coverage information for some subset of test tasks and
uses the information of which files were changed by the developer to run only a subset of tests. This intuitive 
approach aims to select \emph{modification-traversing} tests \cite{536955}, 
which was one of the first approaches studied by Rothermel and Harrold, and widely studied by many others.
The approaches differ in how a test is determined to be \emph{modification-traversing}, 
as some approaches use Control Flow Graphs \cite{366926}, others use execution traces (Vokolos and Frankl in \cite{Vokolos1997PythiaAR})
and some others use coverage information. Like the work of Beszédes et. al. \cite{Beszdes2012CodeCR}, where they
describe populating a coverage database for the C++ components of the WebKit project
and identify the changed modules from a given revision.

For the techniques evaluated in this project, other literature reviews like \cite{Pan2021TestCS} by Pan, et. al.,
present more similar techniques, as they give an overview of Selection and Prioritization techniques
using Machine Learning (ML). In this study, they classify the approaches into four groups: Supervised
Learning, Unsupervised Learning, Reinforcement Learning, and Natural Language Processing-based.

In our project we focus on Supervised Learning techniques, as found in the work by Bertolino et. al. in 
\cite{Bertolino2020LearningtoRankVR}; in their work, they build upon previous work by Spieker et. al. in \cite{DBLP:journals/corr/abs-1811-04122}
by comparing Reinforcement Learning approaches to Supervised Learning approaches using ranking algorithms.
In these two papers, different features are used to characterize test executions, in \cite{DBLP:journals/corr/abs-1811-04122} 
they only use test history information and duration, and in \cite{Bertolino2020LearningtoRankVR} they use 
features concerning program size, McCabe's cyclomatic complexity, object-oriented metrics, and test history.
In contrast, we propose a different set of features, metrics relevant for the AL DSL, and coverage information
available, which is not considered in these works.

Busjaeger and Xie in \cite{Busjaeger2016LearningFT}, evaluate applying a ranking algorithm, SVM rank, for prioritization 
in the case of Salesforce. We highlight that for the features representing
the changes a developer made, they use coverage information by proposing a coverage score, to 
reduce the impact of outdated coverage information. In contrast, we use and compare different ranking algorithms.

\subsection{The ranking problem in test prioritization}

Previous research has focused on interpreting the problem of test prioritization as a \emph{ranking} 
problem. In this section, we give a brief overview of this problem, and some of the algorithms proposed
as a solution. In particular, we explain the algorithms which were the focus of this project
and how we will interpret the problem of test prioritization as an instance of a ranking problem.

In the context of Information Retrieval, the goal of the ranking problem is to obtain relevant resources
from a potentially large collection of those resources for a given information need (query). 
Ranking algorithms are relevant to different problems, for example in search engines, or 
recommender systems.

An approach that has been the subject of extensive research in recent years is \emph{Learning to rank},
a set of ML techniques with the goal of obtaining a \emph{ranking model} out of
training data. This model is reused when unseen queries are given,
to obtain a similar ranking of the documents as for the training data. 

A \emph{ranking model} is a mathematical model that given $D$ a collection of documents
and a query $q$, it returns an ordered set of every element of $D$. Such ordered set is
sorted according to some relevance criterion.

In the case of CI cycle optimization with Test Prioritization, we interpret the query $q$ as the
change that the developer wants to commit to the target branch. The set of documents $D$ corresponds to
the complete test suite. Our relevance criterion corresponds to sorting the
failing tests first (if any) and the rest of the tests can be sorted through different criteria
like duration or test coverage as we will explain in section \ref{s:method-prioritizing-testruns}.

With this interpretation of the Test Prioritization problem, there is still freedom 
in two aspects: the representation of the query $q$ for a given codebase change, and the 
relevance criterion to use. As part of our experiments, we will consider variations
of these aspects as explained in chapter \ref{s:method}.

We will first describe the different metrics that are used in the ranking literature, and then
give a brief overview of the different ranking algorithms used for completeness.

\subsection{Metrics for the ranking problem}\label{s:bg-rnk-metrics}
Research on ranking algorithms has given a diverse set of metrics to compare and evaluate
rankings proposed by these algorithms. 

A common metric used to evaluate ranking algorithms is the Discounted Cumulative Gain (DCG), which was 
proposed by J\"{a}rvelin, et. al. \cite{10.1145/582415.582418}. 
\texttt{DCG@k} is defined for $k$ the truncation level as:
\begin{align*}
DCG@k = \sum_{i=1}^{k}\frac{2^{l_i}-1}{\log(i+1)}
\end{align*}
Where $l_i$ is the relevance given to the $i$-th result. We can see that given a ranking to evaluate, this metric increases
when the first values of the ranking are given a high relevance. In contrast, high relevance values encountered later are 
penalized by $\log(i+1)$.

The truncation level just limits the considered documents for this metric.

\texttt{NDCG@k} is the Normalized version of \texttt{DCG@k}. This metric compares the obtained ranking against
the ideal ranking for that query and computes its corresponding \texttt{DCG@k} denoted by $IDCG@k$:

\begin{align*}
NDCG@k = \frac{DCG@k}{IDCG@k}
\end{align*}

Another metric is the Mean Average Precision (MAP) \cite{zhumap}, which is based on binary classification metrics. Traditionally,
precision and recall are widely used for binary classification. In the context of
Information Retrieval, given a query, \emph{precision} refers to how many documents predicted as relevant
are labeled as relevant, \emph{recall} refers to how many of the documents were correctly
classified from all the documents labeled as relevant.

The average precision ($\text{AveP}$) represents the area under the curve of a precision-recall
plot when considering the first ranked elements:
\begin{align*}
\text{AveP} = \sum_{k=1}^{n} P(k)|R(k)-R(k-1)|
\end{align*}

Where $P(k), R(k)$ are the precision and recall obtained for the first $k$ results.

While our proposed prioritizations to label the dataset are not binary,
they can be considered binary by giving a cutoff point for the assigned relevance.

The Expected Reciprocal Rank (\texttt{ERR@k}) metric, was proposed in \cite{10.1145/1645953.1646033}
by Chapelle, et. al. .It is designed to take the relative ordering between ranked results into consideration.
In contrast to \texttt{DCG}, it does not give the same gain and discount to a fixed position.
It is defined by:

\begin{align*}
\sum_{r=1}^k \frac{1}{r} R_r \prod_{i=1}^{r-1}(1-R_i)
\end{align*}

Where $R_i$ is defined as:

\begin{align*}
R_i = \frac{2^{l_i}-1}{2^{l_m}}
\end{align*}

Where $l_m$ denotes the maximum relevance value of the ranking, and $l_i$ is the relevance given to the $i$-th result.

While this is the least intuitive of the metrics, it correlates better with empiric measurements of search engine applications\cite{10.1145/1645953.1646033}.
It is based on modeling the probability of a user finding its query at a given document position.

As part of our experiments described in chapter \ref{s:method}, we evaluate different metrics
used to train the ranking algorithms. 

\subsection{\emph{Learning to Rank} algorithms}\label{s:bg-tsp-ltr-algs}

We will now briefly describe the different ranking algorithms explored for this project.

\subsubsection{Coordinate Ascent}
Coordinate Ascent is a general \textit{optimization} method, it is based on iterations defined
by maximizing the given function $f$ when fixing all coordinates but one. Formally, the $(k+1)$-th iteration,
has as $i$-th component:
\begin{align*}
x^{k+1}_i = \arg \max_{t\in\mathbb{R}} f(x^{k}_1, ..., x^{k}_{i-1}, t, x^{k}_{i+1}, ..., x^k_n)
\end{align*}

This method has similar convergence conditions as gradient descent. It was first proposed as a
ranking method by Metzler and Croft in \cite{Metzler2006LinearFM}, and it has successfully been applied
to different ranking problems. We give a short overview of how this optimization method is used as
a ranking method, but we refer to the above-mentioned article for a detailed explanation.

The optimization is used for maximizing the different ranking evaluation metrics presented in section \ref{s:bg-rnk-metrics}.
Metzler, et. al. propose a linear model to predict the score of each document and add constraints to the parameter space being optimized.

To make it more concrete, the ranking is induced by a scoring function $S$, for a document $D$ and query $Q$
 of the following form:

\begin{align*}
S(D; Q) = \Omega \cdot f(D, Q) + Z
\end{align*}

Where $f(D, Q)$ denotes the feature vector for such document and query, $Z$ denotes a fixed constant, and $\Omega$
denotes the free parameters to be optimized. In the ranking library used for the implementation in this project, called Ranklib, $Z$ is set to zero.


\subsubsection{MART}
Regression algorithms using gradient boosting were first proposed by Friedman in \cite{Friedman2001GreedyFA},
Multiple Additive Regression Trees (MART) is a gradient boosting technique further developed by Friedman and
Meulman in \cite{Friedman2003MultipleAR}.

As a general introduction, we provide a high-level overview of the method. We refer the reader to \cite{Friedman2001GreedyFA} 
for a more detailed explanation.

In general, this is a regression technique that approximates a function by minimizing some related loss 
function. The idea is to use a linear combination of $M$ different models called weak learners:

\begin{align*}
F(x) = \sum_{i=1}^M \gamma_i h_i(x)
\end{align*}

Where $h_i$ denotes each of the weak learners, and $\gamma_i$ are constants found during training.

The idea is to fit a regression tree to approximate the target function and use the next regression tree to
approximate the residuals of the first. Afterwards, greedily compute a scale ($\gamma_i$ above) for the weak learner that minimizes
the loss function.

These residuals approximate the gradient of the loss function, effectively making this method follow the same
rationale as gradient descent to minimize the loss function.

Regression algorithms such as MART can be used to rank
by regressing a relevance function for each of the documents to rank, minimizing some of the 
different training metrics presented in section \ref{s:bg-rnk-metrics}.

Methods for ranking induced by regression algorithms such as MART are called \emph{pointwise} ranking algorithms
because they only consider the information of a single document to determine its relevance.

\subsubsection{LambdaMART}
LambdaMART\cite{lambdamart} takes its name from its constituents: LambdaRank and MART. In the previous section, we 
explained how MART works for general regression and ranking.

In contrast to \emph{pointwise} algorithms that only consider a single document to assign its relevance,
\emph{pairwise} algorithms consider the relative ordering of pairs of documents. In the case of LambdaMART, 
the aim is to produce a comparison function between documents. With a comparison function, we can obtain the
ranking by sorting.

For the case of this family of algorithms, the aim is to obtain a function that given two documents $x_i$, $x_j$, 
obtains the probability that for a given query $q$, $x_i$ is ranked with higher than $x_j$: $P_{ij}$. With 
such a comparison function, sorting of the complete set of features can be performed.

To do so, the trained model is a function $f$ that only takes as input a feature $x_i$, 
and outputs a real value $f(x_i)$. To obtain the probability of the pairwise comparison a logistic function is used:

\begin{align*}
P_{ij} = \frac{1}{1+e^{-\sigma(f(x_i)-f(x_j))}}
\end{align*}

Where $\sigma$ denotes the logistic growth rate. The loss function used is the cross-entropy measure. 
Minimizing through gradient descent is the idea behind predecessors
of this algorithm like RankNet and LambdaRank.

In LambdaMART this gradient is not computed but predicted by boosted regression trees. We refer
to \cite{lambdamart} for a more thorough explanation of how these gradients are reduced to scalars
subject to optimization by weak learners.

The method has been successfully applied in diverse ranking applications, and in particular, it performed
the best in the analysis given by Bertolino, et. al. in \cite{Bertolino2020LearningtoRankVR} on Test Prioritization.

\subsubsection{RankBoost}\label{s:bg-tsp-rankboost}

Rankboost was proposed by Freund, Yoav, et. al. \cite{10.5555/945365.964285}. Similar to MART, it uses \emph{boosting}, combining
several \emph{weak} learners into a single model.

Each of these learners predicts a relevance function, and therefore a ranking. For training, a distribution $D$ over
$X\times X$ is required, where $X$ is the documents on a query. For this reason, this method is $O(|X|^2)$ in memory, which
can restrict the applicability of this algorithm. 

The distribution $D$ represents which pairs of documents are more relevant to order correctly with regards to
optimizing ranking evaluation metrics. Each learner updates the distribution $D$, emphasizing the pairs that are more relevant for the algorithm
to properly order in that iteration. The final relevance function then becomes a linear combination of each of these learners.
We refer the reader to \cite{10.5555/945365.964285} for more details and further reading.

\subsubsection{AdaRank}
This algorithm was proposed by Xu and Li in \cite{xuliadarank}, it was designed to directly minimize Information Retrieval (IR) performance measures
like \texttt{MAP} and \texttt{NDCG}. It is based on AdaBoost, a binary classifier also based on \emph{boosting},
obtaining a model from the linear combination of several \emph{weak} learners.

On each iteration, it maintains a distribution of weights assigned to each of the training queries.
Such distribution is updated, increasing the values of the queries that the weak learner ranks the worst.
In this way, subsequent learners can focus on those queries for the next rounds.

The weak learners proposed in \cite{xuliadarank} are linear combinations of the metrics to minimize the weight distribution.

\subsection{Metrics for evaluating Test Selection and Prioritization}
\label{sec:bg-metrics-tsp}

While the problem of ranking has been widely studied and metrics have been proposed for evaluating it; it is more
meaningful for our purposes to evaluate the resulting prioritization with metrics in the context of regression testing.

In this section we expand upon some of the metrics previously proposed in the literature, to evaluate the problems of Test Selection
and Prioritization.

\subsubsection{Test selection execution time}
For evaluating selection algorithms, a natural approach is to measure the time it takes to run the subset. To make 
this metric test suite independent a ratio is used:
\begin{align*}
    t_x = \frac{t_S}{t_C}
\end{align*}

Where $t_S$ is the time taken to execute the selection and $t_C$ is the time taken to execute the complete test suite.
Ideally, the aim is to obtain values close to zero for this metric, as this represents a significant improvement
in execution time. In contrast, values close to one represent a similar execution time as the complete test suite.

\subsubsection{Inclusiveness}
In test selection, we do not only rely on the execution time for evaluation. Consider an arbitrarily small subset selected, it would yield good results,
but potentially it could also miss some fault-revealing test cases.

To consider this, inclusiveness is introduced:
\begin{align*}
i = \frac{|S_F|}{|T_F|}
\end{align*}

Where $S_F$ is the set of fault-revealing test cases from a selection, and $T_F$ is the set of test faults in the complete
test suite. For completeness, we define $i=1$ when there are no test faults in the given change.

A \emph{safe} test selection algorithm \cite{366926} always has $i = 1$. As every fault-revealing test is included in the selection.

\subsubsection{Selection size}
On the other hand, high inclusiveness could also be a sign of over-selecting test cases. For example, selecting the whole test
suite trivially has $i=1$. To have a measure of how big the selection is, \emph{selection size} is defined as:

\begin{align*}
    ss = \frac{|S|}{|T|}
\end{align*}

Where $S$ denotes the set of selected tests and $T$ denotes the complete set of tests.
A good selection algorithm strives for having a small selection size, while high inclusiveness.

\subsubsection{Time to the first failure}
Likewise, another time-related metric of interest for prioritization is the time it takes to reach the first failure:
\begin{align*}
    t_{ff} = \frac{t_F}{t_C}
\end{align*}
Where $t_F$ is the time taken to reach the first failure for the proposed prioritization.

\subsubsection{Normalized Average of the Percentage of Faults Detected}
For prioritization, only focusing on time to get to the first failure is skewed. As one could have detected the first failure
soon while prioritizing the rest of the failing test cases with low priority.

To overcome this, a widely used metric, proposed by Elbaum, Malishevsky, and Rothermel in \cite{elbaum2002} is the 
Average of the Percentage of Faults Detected (APFD).

Normalizing this metric to the number of failures detected allows considering cases where no
selected test case was failing. This is useful for evaluating prioritizations that had previously some selection criteria applied.

It is defined by:
\begin{align*}
NAPFD = p - \frac{\sum_{i=1}^mTF_i}{nm} + \frac{p}{2n}
\end{align*}

Where $p$ denotes the ratio of detected faults on a selection against the total amount of faults,
$n$ denotes the number of test cases, $m$ denotes the total number of faults, and $TF_i$ denotes the ranking of
the test case that revealed the $i$-th failure.

This metric represents the proportion of the test failures detected against each executed test.
We aim for this value to be close to 1, representing that the accumulated amount of 
failures detected is obtained early with the prioritization.