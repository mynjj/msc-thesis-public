\chapter{Method}\label{s:method}
The main steps required to use \emph{Learning to Rank} approaches in our context are:
\begin{itemize}
    \item Collect CI jobs, along with information on the tests executed. Since we are also using coverage information, we also need to collect this information.
    \item Obtain vectors representing each test execution for each change to the codebase. These vectors represent the \emph{queries} in the context of \emph{Learning to Rank}.
    \item Label the training dataset by assigning rankings to each. These labels should reflect prioritizing failing tests, the remaining tests could be prioritized with different criteria.
    \item Train the models with the different parameters that could be considered for each. This training is performed with the training dataset produced after representing each query and assigning priorities to them.
    \item Use the models to obtain rankings for the validation dataset. With these rankings, we can obtain the different evaluation metrics that we require to answer our research questions.
\end{itemize}

In the remainder of this chapter, we discuss more details of these steps.

\section{Collecting the dataset of CI executions}\label{s:method-collecting-dataset}

As the initial step of our project, we collect information about the CI executions
from data stored by the DME build system and the history of the repository.

For each CI job, the information extracted was:
\begin{itemize}
    \item Execution model with the tasks that the DME system used as input.
    \item Job execution properties: duration, result, and date.
    \item For each of the tasks executed by the job, information on properties: duration, result, and date.
    \item Other meta information to identify the job in the VCS.
    \item For each of the \emph{application test} tasks, the result of each procedure run for each of the test units, along with information on the duration of its execution.
    \item Comparison with the last merge from the target branch: path and directories where changes were made, type of changes performed, and the content of modified files.
\end{itemize}

The aim was to collect enough properties to represent the changes a developer made, 
along with data to evaluate the prioritized tests.

\subsection{Coverage information for test runners}\label{s:method-collecting-coverage}

The information listed in the previous section was collected from real operation data
of the pipeline. As explained in section \ref{sec:app-tests-al}, there are two different
implementations of test runners that the tasks may use.

As explained in section \ref{s:bg-bc-coverage}, in the current pipeline, tests run with the \emph{CAL test runner}
do have coverage information. However, that is not the case for tests that use the \emph{AL test runner}.

A complete coverage report for all the application tests was not initially available. However, as 
part of this project, modifications to the \emph{AL test runner} were done to allow 
for collecting the same kind of information\footnote{The changes done were based on 
previous work by Nikola Kukrika (nikolak@microsoft.com)}. However, these changes were not 
integrated into the pipeline. Instead, the changes made to the test runner were run against snapshots of 
the codebase in a given time.

It has been discussed previously in research how coverage information may be
outdated and hard to maintain \cite{Bertolino2020LearningtoRankVR}. This is 
partly true in our case as well, however, we acknowledge that the information 
given by coverage can be valuable for our problem.

In \cite{Busjaeger2016LearningFT} a more robust approach to using coverage information is proposed, by defining
a coverage score. A feature like coverage score mitigates the lack of accuracy 
of the coverage information. As additional mitigation to this problem we introduce
windows to compute such coverage scores as it will be shown in section \ref{s:method-characterizing-testruns}.

\subsection{The collection process}

As a general overview, over roughly a week period, real CI jobs in this pipeline were collected. 
Sporadically between these jobs, custom jobs were executed with the required changes
to the \emph{AL test runner} to collect coverage information. 

For the results presented in this work, two of these coverage collecting jobs were 
executed and retrieved, and 172 CI jobs were collected.

Whenever coverage information is required to compute the features of a given CI job,
the coverage information used will be the closest earlier collected one.

The scale of the collected information limited the number of jobs we were able to
collect.\footnote{The information on CI job executions amounted to 6GB of data and 
the coverage data to 44GB. }
\section{Representing tests and codebase changes by a vector}\label{s:method-characterizing-testruns}

For each of the collected CI jobs, we obtain feature vectors, representing
both: the changes made by the developer, and each of the executed tests.

Each of the features in this vector refers to different properties, for which
we consider the following three categories:

\begin{itemize}
    \item Properties of the change to the AL codebase: Properties related to the changes of the AL files in the codebase.
    \item Test history properties: Properties related to the historical behavior of the test.
    \item Coverage properties: Properties related to the coverage information of each test and the change considered.
\end{itemize}

In the rest of this section, we list the properties considered in each category.
In section \ref{s:future-evalp} we discuss some of the different features that 
future work could consider found in the literature.

\subsection{Properties of the change to the AL codebase}

We use the following quantities to represent the changes done to the
AL codebase:

\begin{itemize}
    \item Number of new AL tables.
    \item Number of new AL objects.
    \item Number of modified AL objects.
    \item Number of removed AL objects.
    \item Number of changed tests.
    \item Number of added lines to AL objects.
    \item Number of removed lines from AL objects.
    \item Number of changed files that are not AL objects.
\end{itemize}

These quantities represent numerically the typical changes experienced in the AL DSL. These properties are analogous to
the \emph{Program size} and \emph{Object-oriented} properties used in literature \cite{Bertolino2020LearningtoRankVR}\cite{Busjaeger2016LearningFT},
which are not directly applicable to our case.

\subsection{Test history properties}

For each of the tests in a job we add the following historic properties:

\begin{itemize}
    \item Proportion of times the test has failed within the available data.
    \item From the past $k$ job executions, the proportion of times the test has failed.
    \item Average duration of the test in previous executions.
\end{itemize}

\subsection{Coverage properties}
To understand the effect that coverage information has (\textbf{RQ3}), we also consider coverage properties
to represent the changes done to the codebase for each test.

Using the most recent coverage information collected previous to the given job, we compute:
\begin{itemize}
    \item Ratio between lines covered by that test and the average.
    \item The number of changed files that were covered by that test.
    \item The number of changed lines that were covered by that test within different \emph{windows}.
\end{itemize}

The lines changed were not matched exactly but by \emph{windows}. If a change on a
line nearby was covered, it was counted for such properties. We added features for
different window sizes.

The aim of such is to reduce the impact of having outdated coverage information.
\section{Prioritizing Test Executions in a CI job}\label{s:method-prioritizing-testruns}

As part of our training dataset for the ranking algorithms, we need to associate a priority value
to each of the tests executed in a job. This relevance induces the ranking that we desire the algorithm to 
learn. This is the process commonly referred to as \emph{dataset labeling} in the context of ML.

It is worth emphasizing that this prioritization is not the one against which the results
will be evaluated, as this would be biased. The evaluation will be given only by the 
TSP metrics presented in section \ref{sec:bg-metrics-tsp}.

A ranking can be defined by a priority function (also called relevance function): when each 
test case is assigned a priority, this value can be used to produce a ranking where such 
priority values are in increasing order.

We explain the two different approaches taken to assign priority functions. We created training datasets with both of these priority functions.

\subsection{Failure and Duration decreasing exponential priority}

In \cite{Bertolino2020LearningtoRankVR} Bertolino, et. al. propose a priority function for each test case, based on its duration and outcome.
They define a score for the $i$-th test case $R_i$ by:
\begin{align*}
R_i = F_i + e^{-T_i}
\end{align*}

Where $F_i = 1$ if the test fails, $0$ otherwise, and $T_i$ the duration of its execution.

By design, this score ranks first failing tests and breaks ties via their execution duration.
Furthermore, by this being an exponentially decreasing function, changes
in duration have a larger effect for tests with small duration, than for test executions with
large duration.

\subsection{Coverage discrete priorities}\label{s:method-prioritizingtestcases}
We propose a discrete priority function using coverage information. 

Given the $i$-th test, we define $L_i$ as the number of lines covered by this test.

For a given job executing a subset of tests $\tau$, we define $\mu_{L,\tau}$ to be the mean of
all values $\{L_i\}_{i \in \tau}$. For such job, our proposed priority function prioritizes tests
in the following order:

\begin{itemize}
    \item Failing Tests
    \item New or modified tests in the job
    \item Tests that have no coverage information
    \item Tests covering changed lines, where $L_i \ge \mu_{L,\tau}$
    \item Tests covering changed lines, where $L_i < \mu_{L,\tau}$
    \item Tests where $L_i \ge \mu_{L,\tau}$
    \item Tests where $L_i < \mu_{L,\tau}$
\end{itemize}

Intuitively, this prioritization ranks first failing tests, and then as a conservative 
approach the modified tests in the change, and tests for which we do not have any coverage 
information, for example, new tests. Afterwards, we assign a higher priority to tests that
 traverse the changed lines on their execution, and finally the other unrelated tests.

It is worth reminding the reader, that from such a job we do not have the exact coverage information
but the most recent, previously collected. As we described in section \ref{s:method-collecting-dataset}.

\section{Training ranking models}\label{s:method-training-models}

For each training performed, we varied the training metrics presented in section \ref{s:bg-rnk-metrics}. 

Additionally, for algorithms based on regression trees like MART and LambdaMART, we varied
the number of trees for gradient boosting.

As explained in previous sections, we produced different datasets changing the prioritization
criteria and the features that describe each change. As means of identification for evaluation, 
we name them as described in table \ref{f:table-naming-datasets}.

\begin{table}[h!]
    \centering
    {\renewcommand{\arraystretch}{2.5}
    \begin{tabular}{|c|c|c|}
        \hline
        \textbf{Dataset name} & \textbf{Prioritization} & \textbf{Features of each test execution} \\
        \hline
        \parbox{0.12\textwidth}{\texttt{EP-NCI}} & \parbox{0.30\textwidth}{Decreasing exponential as described in section \ref{s:method-prio-exp}} & \parbox{0.40\textwidth}{
            \begin{itemize}
                \item Properties related to AL changes
                \item Test history properties
            \end{itemize}
        } \\
        \hline
        \parbox{0.12\textwidth}{\texttt{EP-CI}} & \parbox{0.30\textwidth}{Decreasing exponential as described in section \ref{s:method-prio-exp}} & \parbox{0.40\textwidth}{
            \begin{itemize}
                \item Properties related to AL changes
                \item Test history properties
                \item Coverage properties
            \end{itemize}
        } \\
        \hline
        \parbox{0.12\textwidth}{\texttt{CP-NCI}} & \parbox{0.30\textwidth}{Coverage based as described in section \ref{s:method-prioritizingtestcases}} & \parbox{0.40\textwidth}{
            \begin{itemize}
                \item Properties related to AL changes
                \item Test history properties
            \end{itemize}
        } \\
        \hline
        \parbox{0.12\textwidth}{\texttt{CP-CI}} & \parbox{0.30\textwidth}{Coverage based as described in section \ref{s:method-prioritizingtestcases}} & \parbox{0.40\textwidth}{
            \begin{itemize}
                \item Properties related to AL changes
                \item Test history properties
                \item Coverage properties
            \end{itemize}
        } \\
        \hline
    \end{tabular} }
    \caption{Naming of the different datasets created.}
    \label{f:table-naming-datasets}
\end{table}

We considered these different combinations of prioritization criteria and features, in accordance with
the research question \textbf{RQ3}, as we aim to understand the effect that coverage information has 
when applying \emph{Learning to Rank} techniques. Therefore we create datasets with and without coverage
information to compare their effectiveness.
