\section{Representing tests and codebase changes by a vector}\label{s:method-characterizing-testruns}

For each of the collected CI jobs, we obtain feature vectors, representing
both: the changes made by the developer, and each of the executed tests.

Each of the features in this vector refers to different properties, for which
we consider the following three categories:

\begin{itemize}
    \item Properties of the change to the AL codebase: Properties related to the changes of the AL files in the codebase.
    \item Test history properties: Properties related to the historical behavior of the test.
    \item Coverage properties: Properties related to the coverage information of each test and the change considered.
\end{itemize}

In the rest of this section, we list the properties considered in each category.
In section \ref{s:future-evalp} we discuss some of the different features that 
future work could consider found in the literature.

\subsection{Properties of the change to the AL codebase}

We use the following quantities to represent the changes done to the
AL codebase:

\begin{itemize}
    \item Number of new AL tables.
    \item Number of new AL objects.
    \item Number of modified AL objects.
    \item Number of removed AL objects.
    \item Number of changed tests.
    \item Number of added lines to AL objects.
    \item Number of removed lines from AL objects.
    \item Number of changed files that are not AL objects.
\end{itemize}

These quantities represent numerically the typical changes experienced in the AL DSL. These properties are analogous to
the \emph{Program size} and \emph{Object-oriented} properties used in literature \cite{Bertolino2020LearningtoRankVR}\cite{Busjaeger2016LearningFT},
which are not directly applicable to our case.

\subsection{Test history properties}

For each of the tests in a job we add the following historic properties:

\begin{itemize}
    \item Proportion of times the test has failed within the available data.
    \item From the past $k$ job executions, the proportion of times the test has failed.
    \item Average duration of the test in previous executions.
\end{itemize}

\subsection{Coverage properties}
To understand the effect that coverage information has (\textbf{RQ3}), we also consider coverage properties
to represent the changes done to the codebase for each test.

Using the most recent coverage information collected previous to the given job, we compute:
\begin{itemize}
    \item Ratio between lines covered by that test and the average.
    \item The number of changed files that were covered by that test.
    \item The number of changed lines that were covered by that test within different \emph{windows}.
\end{itemize}

The lines changed were not matched exactly but by \emph{windows}. If a change on a
line nearby was covered, it was counted for such properties. We added features for
different window sizes.

The aim of such is to reduce the impact of having outdated coverage information.