\section{Characterizing Test Executions for a Codebase Change}\label{s:method-characterizing-testruns}

For each of the collected CI jobs, we obtain features to characterize
the changes made by the developer, for each of the executed tests in the job.

In this section we explain the different kinds of information obtained to
characterize them.

\subsection{File changes}

We use the following quantities to represent the changes done for each object:

\begin{itemize}
    \item Amount of new AL tables.
    \item Amount of new AL objects.
    \item Amount of modified AL objects.
    \item Amount of removed AL objects.
    \item Amount of changed tests.
    \item Amount of non AL file changes.
    \item Amount of added lines to AL objects.
    \item Amount of removed lines from AL objects.
\end{itemize}

\subsection{Test history properties}

For each of the tests in a job we add the following historic properties:

\begin{itemize}
    \item Proportion of times the test has failed within the available data.
    \item From the past $k$ job executions, proportion of times the test has failed.
    \item Average duration of this test within the available data.
\end{itemize}

\subsection{Coverage properties}

Using the most recent coverage information collected previous to
the given job, we compute:
\begin{itemize}
    \item The proportion of lines covered by that test in relation to the average.
    \item The amount of files changed that were covered by that test.
    \item The amount of lines changed that were covered by that test within different \emph{windows}.
\end{itemize}

The lines changed were not matched exactly, but by \emph{windows}. If a change on a
line nearby was covered, it was counted for such properties. We added features for
different windows sizes.

The aim of such is to reduce the impact of having outdated coverage information.