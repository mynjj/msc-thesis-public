\section{Collecting the dataset of CI executions}\label{s:method-collecting-dataset}

As the initial step of our project, we collect information about the CI pipeline executions
from data stored by the DME build system and the history of the repository.

For each CI job, the information extracted was:
\begin{itemize}
    \item Execution model with the tasks that the build system used as input.
    \item Job execution properties: duration, result, and date.
    \item For each of the tasks executed by the job, information on properties: duration, result, and date.
    \item Other meta information to identify the job in the Version Control System.
    \item For each of the \emph{application test} tasks, the result of each procedure run for each of the test units, along with information on the duration of its execution.
    \item Comparison with the last merge from the target branch: path and directories where changes were made, type of changes performed, and the content of modified files.
\end{itemize}

The aim was to collect enough properties to represent the changes a developer made, 
along with data to evaluate the prioritized tests.

\subsection{Coverage information for test runners}\label{s:method-collecting-coverage}

The information listed in the previous section was collected from real operation data
of the pipeline. As explained in section \ref{sec:app-tests-al}, there are two different
implementations of test runners that the tasks may use.

As explained in section \ref{s:bg-bc-coverage}, in the current pipeline, tests run with the \emph{CAL test runner}
do have coverage information. However, that is not the case for tests that use the \emph{AL test runner}.

A complete coverage report for all the application tests was not initially available. However, as 
part of this project, modifications to the \emph{AL test runner} were done to allow 
for collecting the same kind of information\footnote{The changes done were based on 
previous work by Nikola Kukrika (nikolak@microsoft.com)}. However, these changes were not 
integrated into the pipeline. Instead, the changes made to the test runner were run against snapshots of 
the codebase in a given time.

It has been discussed previously in research how coverage information may be
outdated and hard to maintain \cite{Bertolino2020LearningtoRankVR}. This is 
partly true in our case as well, however, we acknowledge that the information 
given by coverage can be valuable for our problem.

Busjaeger, et. al. in \cite{Busjaeger2016LearningFT} propose a more robust approach to using coverage information, by defining
a coverage score. A feature like coverage score mitigates the lack of accuracy 
of the coverage information. As additional mitigation to this problem, we introduce
windows to compute such coverage scores as it will be shown in section \ref{s:method-characterizing-testruns}.

\subsection{The collection process}

As a general overview, over some time, real CI jobs in this pipeline were collected. 
Sporadically between these jobs, custom jobs were submitted to the build system with the required changes
to the \emph{AL test runner} to collect coverage information. 

In the next sections, whenever coverage information is required to compute properties of a given CI job,
the coverage information used will be the closest earlier collected one.
