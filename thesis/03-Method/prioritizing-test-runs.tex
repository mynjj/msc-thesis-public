\section{Prioritizing Test Executions in a CI job}\label{s:method-prioritizing-testruns}

As part of our training dataset for the ranking algorithms, we need to associate a ranking
to each of the tests executed in a job. This ranking is what we desire the algorithm to 
learn. This is the process commonly referred to as \emph{dataset labeling} in the context of ML.

It is worth emphasizing that this prioritization is not the one against the results
will be evaluated, as this would be biased. The evaluation will be given only by the 
TSP metrics presented in section \ref{sec:bg-metrics-tsp}.

A ranking can be defined by a priority function (also called relevance function): when each 
test case is assigned a priority, this value can be used to produce a ranking where such 
priority values are in increasing order.

We outline two different approaches taken to assign priority functions.

\subsection{Failure and Duration decreasing exponential priority}

In \cite{Bertolino2020LearningtoRankVR} they propose a priority function for each test case, based on its duration and outcome.
They define a score for the $i$-th test case $R_i$ by:
\begin{align*}
R_i = F_i + e^{-T_i}
\end{align*}

Where $F_i = 1$ if the test fails, and $0$ otherwise, and $T_i$ the duration of its execution.

By design, this score ranks first failing tests and breaks ties via their execution duration.
We also notice something else, by this being an exponentially decreasing function, changes
in duration have a larger effect for tests with small duration, than test executions with
large duration.

\subsection{Coverage discrete priorities}\label{s:method-prioritizingtestcases}
We propose a discrete priority function using coverage information. 

Given the $i$-th test, we define $L_i$ as the amount of lines covered by it.

For a given job executing a subset of tests $\tau$, we define $\mu_{L,\tau}$ to be the mean of
all values $\{L_i\}_{i \in \tau}$. For such job, our proposed priority function, prioritizes
in the following order:

\begin{itemize}
    \item Failing Tests
    \item New or modified tests in the job
    \item Tests that have no coverage information
    \item Tests covering changed lines, where $L_i \ge \mu_{L,\tau}$
    \item Tests covering changed lines, where $L_i < \mu_{L,\tau}$
    \item Tests where $L_i \ge \mu_{L,\tau}$
    \item Tests where $L_i < \mu_{L,\tau}$
\end{itemize}

It is worth reminding the reader, that from such given job we do not have the exact coverage information
but the most recent, previously collected. As we described in section \ref{s:method-collecting-dataset}.

We created training datasets with both of these priority functions.