\section{Training ranking models}\label{s:method-training-models}

For each training performed, we varied the training metrics presented in section \ref{s:bg-rnk-metrics}. 

Additionally, for algorithms based on regression trees like MART and LambdaMART, we varied
the number of trees for gradient boosting.

As explained in previous sections, we produced different datasets changing the prioritization
criteria and the features that describe each change. As means of identification for evaluation, 
we name them as described in table \ref{f:table-naming-datasets}.

\begin{table}[h!]
    \centering
    {\renewcommand{\arraystretch}{2.5}
    \begin{tabular}{|c|c|c|}
        \hline
        \textbf{Dataset name} & \textbf{Prioritization} & \textbf{Features of each test execution} \\
        \hline
        \parbox{0.12\textwidth}{\texttt{EP-NCI}} & \parbox{0.30\textwidth}{Decreasing exponential as described in section \ref{s:method-prio-exp}} & \parbox{0.40\textwidth}{
            \begin{itemize}
                \item Properties related to AL changes
                \item Test history properties
            \end{itemize}
        } \\
        \hline
        \parbox{0.12\textwidth}{\texttt{EP-CI}} & \parbox{0.30\textwidth}{Decreasing exponential as described in section \ref{s:method-prio-exp}} & \parbox{0.40\textwidth}{
            \begin{itemize}
                \item Properties related to AL changes
                \item Test history properties
                \item Coverage properties
            \end{itemize}
        } \\
        \hline
        \parbox{0.12\textwidth}{\texttt{CP-NCI}} & \parbox{0.30\textwidth}{Coverage based as described in section \ref{s:method-prioritizingtestcases}} & \parbox{0.40\textwidth}{
            \begin{itemize}
                \item Properties related to AL changes
                \item Test history properties
            \end{itemize}
        } \\
        \hline
        \parbox{0.12\textwidth}{\texttt{CP-CI}} & \parbox{0.30\textwidth}{Coverage based as described in section \ref{s:method-prioritizingtestcases}} & \parbox{0.40\textwidth}{
            \begin{itemize}
                \item Properties related to AL changes
                \item Test history properties
                \item Coverage properties
            \end{itemize}
        } \\
        \hline
    \end{tabular} }
    \caption{Naming of the different datasets created.}
    \label{f:table-naming-datasets}
\end{table}

We considered these different combinations of prioritization criteria and features, in accordance with
the research question \textbf{RQ3}, as we aim to understand the effect that coverage information has 
when applying \emph{Learning to Rank} techniques. Therefore we create datasets with and without coverage
information to compare their effectiveness.
