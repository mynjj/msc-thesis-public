\chapter{Conclusion}

In this project, we created different test prioritization datasets with 
information collected from the CI pipeline of \emph{Business Central}. These datasets included
different features to represent changes to a codebase and criteria to prioritize tests.
With these datasets, different ranking algorithms were trained. For the validation dataset,
the proposed prioritizations were evaluated.

The infrastructure to collect CI history information and related coverage information,
as well as the extraction mechanism of features can be found in the accompanying 
repository of this project. Such tooling can be used to increase the strength and confidence of the 
evaluation here presented by adding more samples and extending 
the time from which the dataset is collected. 

We also provide the ranking datasets used for this project in the 
accompanying repository. In contrast to existing datasets
for the prioritization problem, we include coverage information properties.

We proceed to discuss the different research questions stated in chapter \ref{s:introduction}.

\section{RQ1: Which of the ranking techniques yields the best prioritization results?}
From the evaluated algorithms, we found that the 
best performing and most consistent approach was Coordinate Ascent
with the training metric \texttt{NDCG@30}.

For this algorithm, across the different datasets considered, the average NAPFD value obtained was 82\%.
For the \texttt{CP-NCI} dataset, the average NAPFD value obtained was 86\%.

\section{RQ2: How much could we reduce the number of tests and their execution time?}
For the best configurations of Coordinate Ascent described for \textbf{RQ1},
the induced \emph{safe} selection was 40\% the size of the complete test suite.
This means that for this dataset, the CI pipeline would execute 60\% fewer tests.

The corresponding execution time for the proposed selected tests  is 10\% of the execution time
of the complete test suite.

\section{RQ3: What is the effect of using coverage information for \emph{Learning to Rank} techniques?}

In contrast to existing \emph{Learning to Rank} approaches to the problem,
we use coverage information. We found that the best-performing dataset was \texttt{CP-NCI},
which uses coverage for prioritizing the training dataset, but does not use it as part
of the features that identify each test. 

\texttt{CP-NCI} consistently achieved better results in every algorithm and
configuration. This suggests that for this pipeline there exists a correlation between a good prioritization
in regards to test prioritization metrics and a prioritization that executes lines covered by the changes first. 

However, adding coverage information as part of each test increases the dimensionality. Increasing the dimensions of the input 
space requires more training samples to obtain reliable results. This could explain why \texttt{CP-CI} performed worse than \texttt{CP-NCI}.

\section{Threats to validity}\label{s:conclusion-threats}

Although the results are promising, it is important to consider
the threats to the validity of the evaluation. The most relevant threat 
is a biased dataset.

As explained in section \ref{s:method-collecting-dataset}, the dataset was 
collected over a week. Also, as previously explained, the size of 
the collected dataset and test suite restricted the training of the ranking
algorithms.

Among some of the reasons this could have biased the evaluation, is that
a week is a short time to be representative of the overall test execution dynamics.

For example, a developer working on a feature may re-run the pipeline and continuously
fail the same set of related tests. Given that history features are used, the learning algorithms
could use this as an indicator of a high priority test. Which may not be true after the 
work is done. In an online implementation of these techniques, this problem
could be mitigated by periodic retraining of the model.
%