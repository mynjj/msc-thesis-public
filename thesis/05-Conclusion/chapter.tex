\chapter{Conclusion}

In this project, we created different test prioritization datasets with 
information collected from the CI pipeline of \emph{Business Central}. These datasets included
different features to represent changes to a codebase and criteria to prioritize tests.

With these datasets, different ranking algorithms were trained and evaluated. We
found that the best performing and most consistent approach was Coordinate Ascent
with the training metric \texttt{NDCG@30}. The induced \emph{safe} selection with
this approach reduces the number of tests executed by 60\%, and their execution time 
by 90\%.

In contrast to existing approaches to the problem using \emph{Learning to Rank},
we use coverage information. We found that the best-performing dataset
used coverage for prioritizing the training dataset. 

Existing datasets in research for this problem do not include 
coverage information. The ranking datasets used for this project 
can be found in the accompanying repository. 

The infrastructure to collect CI history information and related coverage information,
as well as the extraction mechanism of features can also be found in the accompanying 
repository of this project. Such tooling can be used to increase the strength and confidence on the 
evaluation here presented by adding more samples and extending 
the time span from which the dataset is collected. 

Having a data model for the changes a developer makes in a codebase increases
the insight that can be obtained for future analysis of this CI pipeline.

\section{Threats to validity}\label{s:conclusion-threats}

Although the results are promising, it is important to consider
the threats to validity of the evaluation. The most relevant threat 
is a biased dataset.

As explained in section \ref{s:method-collecting-dataset}, the dataset was 
collected over a week. Also, as previously explained, the size of 
the collected dataset and test suite restricted the training of the ranking
algorithms.

Among some of the reasons that could have biased the evaluation, is that
a week is a short time to be representative of the overall test execution dynamics.

For example, a developer working on a feature may re-run the pipeline and continuously
fail the same set of related tests. Given that history features are used, the learning algorithms
could use this as an indicator of a high priority test. Which may not be true after the 
work is done. A possible mitigation to this is to continuously retrain the algorithm.

%