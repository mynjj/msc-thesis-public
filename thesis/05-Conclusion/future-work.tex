\section{Future work}

\subsection{Usage of a prioritization technique in context of the existing sytem.}

This work focused on an offline evaluation of the techniques, for an online implementation
other technical challenges and practical considerations remain.

So far in our discussion, we did not tie how proposing a prioritization
with these techniques relates to how they are executed by the DME system explained
in section \ref{s:bc-ci-dme}.

In this section we outline the required changes, and possible strategies to use
such rankings.

Recall that the DME system, executes tasks from a given job, by
traversing each of the required dependencies defined by the model of the job. 
A single task may execute multiple test codeunits or test solutions. 

Given a ranking proposed by these techniques, we can use the induced selection
algorithm to reduce the amount of test codeunits that each of these tasks execute.

Furthermore, if the selection results on \emph{job tasks} with no test codeunits to
execute, we can remove this task, along with the dependencies that were 
only required by this task. By doing so recursively we can remove entire paths of the
job's execution. In general terms, this corresponds to the reachable vertices
of these removed vertices that are not reachable by any other non removed vertex 
on the opposite graph.

For clarity, see figure \ref{f:conc-fut-dag-removingtask}, where a DAG representing the dependencies of a job 
execution is given. Filled with black are the test tasks that after the selection
algorithm, had no tests to execute. Filled with gray are the tasks that were only
required for such tests, and therefore could be removed.

\begin{figure}
    \centering
    \def\svgwidth{0.5\columnwidth}
    \includesvg[inkscapelatex=false]{thesis/figures/network-plots/removing-tasks-model}
    \caption{Tasks and predecessors removed from the job model.}
    \label{f:conc-fut-dag-removingtask}
\end{figure}

This would be an effective use of the selection proposed, however engineering is required
to allow for such dynamic changes of a job's execution. In particular, test codeunits
are not \emph{first-class citizens} on the data model proposed by the DME, as
it is instead responsibility of the task's implementation.

For using the prioritization, other strategies can be leveraged. Since test tasks 
can be performed in parallel by the DME, a proposed prioritization of all the 
tests in the job can not be executed as given. 

Instead, the prioritization can be used locally on each task. That is, for a given
test task, and the overall prioritization, one can get a local prioritization for the
tests belonging to such task. However, the engineering to allow dynamically sorting of the test 
codeunits by the test runners is missing to implement.

Finally, the current build system allows for assigning priorities to the tasks to run,
which is taken into consideration when deciding which task to execute next from the 
set of tasks with completed dependencies. This could be dynamically assigned based on 
the prioritization of the tests being ran. Note that this would also require for the
DME system to have knowledge of the tests being run by a task.

\subsection{Tackling CI optimization from other angles}
The product studied has several CI pipelines that constitute each of the different
development cycles, from different areas that the product has. We studied a single
optimization strategy for one these CI pipelines, namely Test Selection and Prioritization.

However, different strategies exist in the literature, for instance usage of
test suite minimization to remove superfluous tests from a codebase.

Another approach could be to have learning models to predict task failure
of each of the tasks a job is comprised from. A similar feature vector as
obtained for this project, representing the change could be part of the training 
data used to train a binary classifier, for which extensive research exists.

\subsection{Further evaluation of prioritization techniques}
Reducing the amount of tests considered for each job execution for our training
dataset, would allow for a larger timespan to be considered by our evaluation.

This would increase the confidence and validity on the results presented in this project.
