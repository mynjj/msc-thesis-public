\documentclass{thesis/resources/ituthesis}

\usepackage{graphicx}
\usepackage{amsmath}
\usepackage{svg}
\usepackage{calc}
\usepackage{fancyvrb}
\usepackage{multirow}
\usepackage{lscape}
\usepackage{longtable}
\usepackage{hyperref}

\settitle{Testing Faster in\\ Continuous Integration Pipelines. \\An Industrial Case Study}
\setauthor{Diego Joshua Martínez Pineda}
\setsupervisor{Mahsa Varshosaz}
\setdate{May 2022}

\begin{document}

\frontmatter

\thetitlepage
\newpage
\chapter{\texttt{SYN/ACK}$^1$}
\footnotetext[1]{A TCP joke, because less people would get a UDP one.\footnotemark[2]}
\footnotetext[2]{Yes, that was another joke.}

\parbox{0.8\textwidth}{
Oh... so many things to acknowledge and be thankful about. It transcends words really.

Thanks to the kind people and great engineers at Microsoft, it is an honor to be able to learn
from all of you. Thanks to the team behind the ITU's HPC, always helpful and effective. 
And thanks to Mahsa, my supervisor, for her extensive proofreading, and guidance throughout 
the project, I truly appreciate it. 
}

\vspace{3.5cm}

\parbox{\textwidth}{
    \begin{flushright}
``Gracias" no son suficientes para las personas que me han ayudado de la manera 
m\'as profunda. Gracias a mis abuelos Abel y Ana Mar\'ia, por su amor y sabidur\'ia.
Gracias a Sof\'ia, Elisa, Miguel y Anabelle, la familia que me ha hecho quien soy y 
que me ha ense\~nado lo que significa dar desinteresadamente. 

Finalmente, gracias a la familia que me ha dado vital apoyo en esta etapa de mi vida. Gracias a 
mis primos Karla y Oskar, a mi t\'ia Patricia por toda su entrega, amor y paciencia. A mi t\'io, Anders S\o{}rensen, 
osito, que me ense\~n\'o tanto en tan poco tiempo. Continuo aprendiendo de ti, te admiro, extra\~no 
y llevo en mi coraz\'on por el resto de mi vida.  

M\'as alla de las palabras, los amo. Todos mis logros son en verdad suyos.
    \end{flushright}
}

\newpage
\thispagestyle{empty}
\begin{abstract}
In this project, we present the evaluation of test selection and prioritization techniques, with data 
collected from a Continuous Integration (CI) pipeline of the Microsoft \emph{Business Central} project.

Having an effective technique for test prioritization in a project is relevant in Continuous Integration
pipelines. Their execution can be time-consuming and energy inefficient. Furthermore, these are typically executed
multiple times per day in a large team. Speed and efficiency not only benefit the development cycles but overall development
costs and energy impact.

We reduce the test prioritization problem to a ranking problem. This is done by extracting 
numeric representations of the changes in a codebase, and using them as input vectors to 
different ranking algorithms. In contrast to existing research, we use different representations;
most notably, the usage of test coverage information. 

Our results show that for the datasets created, using coverage information for assigning priorities 
increases the effectiveness of the prioritization used. However, using coverage information for representing 
changes reduced its effectiveness. 

Based on our results, we argue that with this approach to prioritization we would be able to reduce 
the number of tests that were run by 60\% using a conservative estimate of the best performing algorithm,
which corresponds to an execution time improvement of 90\% in average. This suggests that these approaches
are well suited for the pipeline under study.

Furthermore, we also give an outline of how this prioritization could be used in different stages 
of the existing pipeline.
\end{abstract}

\newpage
\tableofcontents

\mainmatter

\chapter{Introduction}\label{s:introduction}

Large software systems are a composition of intertwined units of functionality,
often related in complex ways. They can be developed across many years and by 
large teams. It is no surprise that complexity arises quite easily. 

It becomes progressively harder to add new features and improvements while being 
sure that everything else works as it should. That is, the software remains
correct. To tackle this problem of complexity, many strategies and different 
angles have been studied and implemented by large-scale software systems. 

Automated regression testing \cite{baresiregtest} is one of the main strategies widely adopted by the industry \cite{10.1007/s10606-009-9098-7},
to ensure that changes to the code will not negatively affect the users of the system. 
This technique consists of developers writing automated tests: code that executes scenarios that a user would experience
and asserts that the desired properties hold. Afterwards, when changes to the code are done,
executing successfully the set of automated tests gives confidence that such scenarios will
not be impacted by the change. In practice, this is an effective technique to preserve the quality of the system while
evolving. \cite{10.1007/s10606-009-9098-7}\cite{baresiregtest}

Another challenge in software engineering consists in having an effective development process when
working with large teams. One of the solutions widely adopted by the industry is to use \emph{Continuous Integration} (CI) processes.

The main idea of the CI software development process is that developers should 
integrate code changes frequently. When developers integrate their changes, 
they make other developers on the team aware of these changes. If the changes
 conflict with their work, they can react accordingly.

Frequently integrating changes to the product increases the risk of making mistakes.
For that reason, as a safeguard, the set of automated tests should be run before 
integrating every change. The process of building and testing the code automatically
every time a change is meant to be integrated is called a CI pipeline and it is typically
enforced in the development processes of industrial software products.

By frequently integrating changes with the regression test suite backing its correctness
we ensure quality in every small step.

However, there is a drawback: as the system evolves, more regression tests
get added, and eventually, it can become time-consuming to run the complete
set of automated tests. Even more, if this is done every time a developer wishes to
integrate changes to the codebase, which is the case in the context of CI.

To address this problem, academic research and industry have focused on several angles.
Algorithms on Test Selection and Prioritization (TSP) are techniques that aim to 
automatically propose a relevant sorted subset of tests to execute. The rationale
behind these techniques is that since the goal of executing the test suite is to
find failing test cases for the given change, we can avoid executing unrelated tests
by selecting a subset of them and executing first the ones that are more likely to fail.

Not only is optimizing CI pipelines relevant for improving the agility of software development, 
but it also has a positive impact on the energy resources they require.
These pipelines are run multiple times a day, and the cost of operating the infrastructure
required to keep the system evolving is significant.

For this project, TSP techniques are applied and evaluated for data collected from one of the CI pipelines
of the \emph{Business Central} project. \emph{Business Central} is an Enterprise Resource Planning (ERP)
software product by Microsoft.  The aim is to find a suitable technique for this pipeline and to 
propose possible directions to improve the CI cycle feedback time and effectiveness.

The techniques explored are based upon previous work by Bertolino
et. al. in \cite{Bertolino2020LearningtoRankVR}, where the prioritization problem is transformed
into a \emph{ranking} problem. We then apply different \emph{Learning to Rank} algorithms and evaluate
their results. However, in contrast to their approach, we propose the usage of coverage information 
as a criterion for these techniques.

For clarity, we will focus on the following research questions:
\begin{itemize}
    \item \textbf{RQ1}: Which of the ranking techniques yields the best prioritization results?
    \item \textbf{RQ2}: To what extent the number of tests and their execution time can be reduced?
    \item \textbf{RQ3}: What is the effect of using coverage information for \emph{Learning to Rank} techniques?
\end{itemize}

Throughout this work, we also outline the different challenges and technical difficulties that adapting
academic techniques posed when taking into consideration the practical realities
of an industrial environment. Additionally, we discuss how these techniques could
be used in the existing context of the pipeline.

All the code for the different stages of the project can be found 
in the accompanying repository.\footnote{\url{https://github.com/mynjj/msc-thesis-public}}%
\chapter{Background}
\section{Business Central}

\emph{Business Central} (BC) is an Enterprise Resource Planning (ERP) software 
system targeting Small and Midsized Businesses (SMB) from Microsoft. Its functionality
spans several areas of a company such as finance, sales, warehouse management, and
many others.

BC allows for Microsoft partners to provide custom extensions that suit customer
needs as much as required. This is done through application extensions that developers
can write in AL: a Domain Specific Language (DSL) for the application logic. 
These extensions modify the experience end-customers have with the product and 
enhance the product with any custom logic their business may require.
Not only partners provide extensions for BC, all the business logic
are first-party extensions that Microsoft developers maintain for the core business 
functionality of the product.

In this thesis project, we will focus on the CI pipeline and regression tests
used for the business logic code (also referred to as \emph{application} code).
The project has several other pipelines for the different parts of the product,
but for the scope of the project, we focus on optimizing the CI pipeline through
techniques for test selection and prioritization targeting changes in the AL DSL.

\subsection{The \emph{application} CI pipeline and the DME system}\label{s:bc-ci-dme}

The BC \emph{application} code follows a traditional CI cycle for development, which we describe 
for completeness and clarity. Whenever developers complete tasks by making new 
changes to the \emph{application} code, they create a Pull Request on the 
Version Control System (VCS) for the desired target branch. Before this request
succeeds and the code is merged, certain checks have to be fulfilled, which 
include the execution of automated tests.

To execute automated tests, a \emph{job} is started on the Distributed Model Executor (DME)
system, the internal build system for the product.  A job is defined by a set of 
tasks, defined by scripts that execute the different required stages. Such specification is
called the \emph{model} of the job. This set of tasks may have dependencies between each other. 
We can represent this set of tasks with dependencies as a Directed Acyclic Graph (DAG), where
each task corresponds to a node and an edge corresponds to the target depending on the source.

All these tasks and the job definition are defined within the same
Version Control repository as code, giving the application developers complete 
control over the job's execution. A Pull Request triggering the CI pipeline is not the only way 
a developer can request \emph{jobs}. They can also require them at any point in the development 
cycle for their convenience.

Some of these tasks correspond to running different groups of tests. Therefore, the cost of executing 
certain tests belonging to a task depends on the dependencies the task has.
We explain this by giving the following example: Suppose a \emph{job} being executed 
is defined by a model consisting of the tasks shown in figure \ref{fig:example-dag-tasks} as the 
associated DAG for such a set of task-dependencies. 

\begin{figure}
    \def\svgwidth{\columnwidth}
    \includesvg[inkscapelatex=false]{thesis/figures/network-plots/example-tasks}
    \caption{Example of tasks that depend on each other for a job execution represented as a DAG.}
    \label{fig:example-dag-tasks}
\end{figure}

We can see that the task \texttt{RunApplicationTests} depends on both \texttt{BuildTests}
and \texttt{BuildApplicationTestsDatabase}. On the other hand, the task 
\texttt{RunSystemTests} only depends on \texttt{BuildTests}.

In this example, the DME system could assign two different machines to execute 
\texttt{BuildTests} and \texttt{BuildApplicationTestsDatabase}. Let us suppose that\\
\texttt{BuildApplicationTestsDatabase} takes longer. When the task \texttt{BuildTests}
is done, the task \texttt{RunSystemTests} can already begin as its only dependency
has finished. However, \texttt{RunApplicationTests} will have to wait for its other
dependency to start execution. In this scenario, running tests in the task \texttt{RunApplicationTests}
has a higher cost than running tests in \texttt{RunSystemTests}.

This means that when executing test selection or prioritization considering the complete
set of tests. Some of them entail a higher computing cost depending on the dependencies they
have.

The DAG of task-dependencies for the \emph{Application} pipeline of the \emph{Business Central} project is not as simple 
as the given example. In fact, it is so large that we can not show it meaningfully on a
page. However, to satisfy the reader's curiosity, figure \ref{fig:full-job-metamodel-dag} 
shows a complete diagram for the metamodel of this pipeline.\footnote{A 
metamodel in this context is a specification for the model that the jobs execute, so it is \textit{smaller} and easier to show.}

\begin{figure}
    \centering
    \def\svgscale{0.1}
    \includesvg{thesis/figures/network-plots/metamodel-nonames}
    \caption{Associated DAG of the metamodel of job executions of the \emph{Application} pipeline for the \emph{Business Central} product.}
    \label{fig:full-job-metamodel-dag}
\end{figure}

\subsection{Test Selection currently on BC \emph{application} tests}
\label{sec:bg-bc-test-selection-currently}

In the current CI pipeline of the \emph{application} code of BC, there is 
a selection method based on test coverage information. We will
define more thoroughly the definition of a test selection method in section \ref{s:tsp-tech},
for now, it suffices to think of it as selecting just a subset of tests from the complete
set of tests. In this section, we review how tests are defined for \emph{application} code, and what kind
of coverage information we have available when executing tests.

\subsubsection{Application Tests in AL}\label{sec:app-tests-al}
We first give an overview of how tests are defined in AL, \emph{Business Central}'s DSL for business logic.  
For a more thorough overview of the AL language and its' role in the \emph{Business Central} system, 
we refer to Appendix \ref{sec:appendix-allang}.

AL organizes the code in \emph{objects} \footnote{Note that objects in this context do not 
correspond to objects in the traditional sense of Object-Oriented Programming.}, these
\emph{objects} represent different units of functionality for the different features of the
product, for example, tables in a database, or pages the user can interact with. A common 
type of \emph{object} is a \emph{codeunit} which is conformed by different procedures that
can be called from any other \emph{AL object}\footnote{\emph{Modules} are a similar concept analogous
to \emph{codeunits} in more traditional languages like Python, NodeJS, Rust, Haskell, among others.}.

An \emph{application} test is written also in AL, as a \emph{codeunit} with specific annotations for it to 
be identified by test runner applications. In figure \ref{fig:bg-bc-test-codeunit} you can see
an excerpt of a group of test procedures defined under a test codeunit. From a more general perspective, 
we can think of test codeunits as groups of test scenarios.

\begin{figure}
    \begin{Verbatim}[fontsize=\tiny]
codeunit 135203 "CF Frcst. Azure AI"
{
    Subtype = Test;
    TestPermissions = NonRestrictive;
    // ...
    [Test]
    procedure AzureAINotEnabledError()
    // ...
    [Test]
    procedure NotEnoughHistoricalData()
    // ...
    [Test]
    procedure FillAzureAIDeltaBiggerThanVariancePercNotInserted()
    // ...
}
    \end{Verbatim}
    \caption{Excerpt of the test codeunit 135203.}
    \label{fig:bg-bc-test-codeunit}
\end{figure}

Test runners are also implemented in AL. These are programs that run the test codeunits, with different
settings, like permission sets to use, or whether or not the test runner should persist changes after 
test execution, which can be desirable according to the scenarios the developer wishes to test. Currently, 
the project has two different implementations of test runners, used for different
sets of tests, a result of historical legacy. For clarity, we distinguish them
as \emph{CAL Test Runner} and \emph{AL Test Runner}.

In the current \emph{application} CI pipeline, an \emph{Application Test} task on the job model can use either of these 
two test runners depending on their needs, or historical legacy. In the current model definition,
roughly half of the test tasks are using \emph{CAL Test Runner} and the other half the \emph{AL Test Runner}.

Aside from historic differences, for our purposes, we highlight a significant
difference: The \emph{CAL Test Runner} can produce coverage information of test execution, and the \emph{AL Test Runner}
currently does not.

\subsubsection{Coverage information}\label{s:bg-bc-coverage}
AL has implemented a language primitive that can be used to record information about which lines 
of code of which objects were executed. In a broad sense, this is how the test coverage information from 
the \emph{CAL Test Runner} is produced. After setting up the context for each test, this primitive is used
to record activity on each AL object throughout execution.

The \emph{CAL Test Runner} currently has 3 different kinds of output relating to coverage information.
We will use information that will allow us to: given a test codeunit executed by the test runner, 
obtain a list of the different \emph{AL objects} and lines within these objects that were run by
the test execution.

For concreteness and to further illustrate the available information, we show an example. In figure \ref{fig:bg-bc-test-codeunit}
we have an excerpt of the test codeunit 135203, containing several test procedures. In figure \ref{fig:bg-bc-coveragefile} we can see
a corresponding excerpt of its coverage file. This coverage information was collected for a given state of the codebase.

Each row on the coverage file corresponds to a line being executed when executing test codeunit 135203. The first 3 columns 
identify each line by giving the Object Type and Object Id the line belongs to, and the Line Number to be considered. The last
column corresponds to the number of times the line was executed. For example, in figure \ref{fig:bg-bc-coveragefile},
we can see that Codeunit 28 had among others lines 111, 112, 115, and 116  executed once. In 
figure \ref{fig:bg-bc-covered-file} we can see such lines.

\begin{figure}
    \begin{Verbatim}[fontsize=\tiny]
"Table","4","17","1"
"Table","4","18","1"
...
"Table","14","112","1"
"Table","14","127","1"
...
"Codeunit","28","111","1"
"Codeunit","28","112","1"
"Codeunit","28","115","1"
"Codeunit","28","116","1"
"Codeunit","28","119","1"
"Codeunit","28","122","1"
...
    \end{Verbatim}
    \caption{Excerpt of the coverage file for the test codeunit 135203. The columns correspond to Object Type, Object Id, Line Number, and Number of Hits}
    \label{fig:bg-bc-coveragefile}
\end{figure}

\begin{figure}
    \begin{Verbatim}[fontsize=\tiny]
1  codeunit 28 "Error Message Management"
...
106    local procedure GetContextRecID(ContextVariant: Variant; ...
107    var
108       RecRef: RecordRef;
109        TableNo: Integer;
110    begin
111        Clear(ContextRecID);
112        case true of
113            ContextVariant.IsRecord:
114                begin
115                    RecRef.GetTable(ContextVariant);
116                    ContextRecID := RecRef.RecordId;
117                end;
    \end{Verbatim}
    \caption{Some of the lines covered by test codeunit 135203.}
    \label{fig:bg-bc-covered-file}
\end{figure}

\subsubsection{Test Selection with coverage information}
Currently, the CI pipeline performs test selection for tasks executed with the \emph{CAL Test Runner},
for which coverage information of each executed test codeunit is recorded.

The selection technique is based on obtaining the lines and files affected
by integrating a developer's change and finding such lines on the recorded coverage information.

For all those matching lines, the corresponding test codeunit is selected for execution. This 
effectively selects every test which traversed during its execution the lines being changed
by the developer.

However, for tests being run by the \emph{AL Test Runner}, this selection is not available as the corresponding 
coverage information is not available. We could classify this selection technique as a partial selection
based on coverage traces.

As we will explain in section \ref{s:method-collecting-coverage}, we will use this coverage 
information, and add features to the \emph{AL Test Runner} to allow for collecting coverage
traces as well.
\section{Test Selection and Prioritization techniques}\label{s:tsp-tech}

Test Selection and Prioritization techniques have a large body of knowledge,
the result of extensive focus from both academy and industry.

In this context, the problem of test selection is defined as: given a change
to the codebase, and a complete set of tests. Obtain a subset of tests,
such that the capacity of fault detection of the test suite is not lost, i. e.
if for such change, the test suite execution results in failure, this subset
of tests should fail as well. A stronger condition is given by a \emph{safe} 
selection algorithm, which requires that every failing test case to be included in the selection.

The problem of test prioritization has the same inputs, and it aims to find a
sorting for the tests to be executed that prioritizes running the tests that
are more likely to fail first.

Intuitively our aim in this problem is that given a change in the codebase,
and a complete test suite to execute. Determine a subset (selection) that
does not miss any fault-revealing test case, and a sorting (prioritization)
that increases the probability of failing first.

It is worth emphasizing that given a prioritization technique, we can induce
selection techniques by selecting the first prioritized tests. The size of
such selection could be determined by a given size, duration, or other criteria.
This is how we obtain the selections evaluated in section \ref{s:results}.

\subsection{Related work}
\label{sec:bg-tsp-related-work}

Yoo and Harman in their survey \cite{Yoo2012RegressionTM} present a detailed overview of techniques in the problems of
regression test selection, prioritization, and minimization\footnote{Minimization is a problem not dealt with in this project, it consists of removing superfluous tests from a test suite}.
 In this survey, several foundational works on this area are presented, formal definitions, and metrics to evaluate this problem.

It is also worth mentioning the similar techniques found in the literature to the already 
existing selection technique in \emph{Business Central}. As explained in section \ref{sec:bg-bc-test-selection-currently}
the CI pipeline of interest of the product collects coverage information for some subset of test tasks.
Using the information of which files were changed by the developer and the coverage information
for each test \emph{codeunit}, a selection is proposed. 

This intuitive approach aims to select \emph{modification-traversing} tests \cite{536955}, 
this was one of the first approaches studied by Rothermel and Harrold, and widely studied by many others.
The approaches differ in how a test is determined to be \emph{modification-traversing}, 
as some approaches use Control Flow Graphs \cite{366926}, others use execution traces (Vokolos and Frankl in \cite{Vokolos1997PythiaAR})
and some others use coverage information. Like the work of Beszédes et. al. \cite{Beszdes2012CodeCR}, where they
describe populating a coverage database for the C++ components of the WebKit project
and identify the changed modules from a given revision.

As a starting point for this project, we reviewed the survey by Pan, et. al. \cite{Pan2021TestCS} as 
such survey focused specifically on Selection and Prioritization and techniques using 
Machine Learning (ML). We initially aimed to find approaches using Reinforcement Learning,
as online learning could be beneficial for this use case. In this area, we find the case of
Speiker et. al. in \cite{DBLP:journals/corr/abs-1811-04122}, where a reinforcement learning agent is proposed using only
history of failure and duration of previous iterations.

This leads us to a publication that was the main inspiration and outlined our approach:
the proposal by Bertolino et. al. in \cite{Bertolino2020LearningtoRankVR}, where they do a
more thorough comparison of the effectiveness of Reinforcement Learning approaches, and
approaches using ranking algorithms.

Another similar approach is given by Busjaeger and Xie in \cite{Busjaeger2016LearningFT}, where they evaluate 
applying a ranking algorithm for prioritization in the case of Salesforce. We highlight that for the features representing
the changes a developer made, they use coverage information. In particular, they propose a coverage score, to 
reduce the impact of outdated coverage information, which is a reality in large scale environments such
as \emph{Business Central}.

\subsection{Ranking algorithms}

As explained in section \ref{sec:bg-tsp-related-work}, previous research has 
focused on interpreting the problem of test prioritization as a \emph{ranking} 
problem. In this section, we give a brief overview of this problem, and the algorithms proposed
for it. In particular, the techniques we will focus on for this project
and how we will interpret the problem of Test Prioritization as an instance
of a ranking problem.

In the context of Information Retrieval, the goal of the ranking problem is to obtain relevant resources
from a potentially large collection of them for a given information need (query). 
Ranking algorithms are relevant to different problems. Examples of systems using them are search engines, or 
recommender systems.

The ranking problem has been extensively studied as it is fundamental for dealing with the
information overload that working with computer systems creates.

An approach that has been the subject of extensive research in recent years is \emph{Learning to rank},
a set of ML techniques whose goal is to obtain a \emph{ranking model} out of
training data. This model is reused when new queries to the system are given,
with the goal of ranking new unseen queries in a similar way as the training data.

A \emph{ranking model} is a mathematical model that given $D$ a collection of documents
and a query $q$, it returns an ordered set of every element of $D$. Such ordered set is
sorted according to some relevance criterion.

In the case of CI cycle optimization with Test Prioritization, we interpret the query $q$ as the
change the developer wants to commit to the target branch. The set of documents $D$ corresponds to
the complete test suite. Our relevance criterion corresponds to sorting the
failing tests first (if any) and the rest of the tests can be sorted through different criteria
like duration or test coverage as we will explain in section \ref{s:method-prioritizing-testruns}.

With this interpretation of the Test Prioritization problem, there is still freedom 
in two aspects: the representation of the query $q$ for a given codebase change, and the 
relevance criterion to use. We will vary these parameters as part of our experiments 
as explained in chapter \ref{s:method}.

We will first describe the different metrics that are used in the ranking literature, and then
give a brief overview of the different ranking algorithms used for completeness.

\subsubsection{Metrics for the ranking problem}\label{s:bg-rnk-metrics}
Research on ranking algorithms has given a diverse set of metrics to compare and evaluate
rankings proposed by these algorithms. Note that these metrics in our context are used for
training and not for evaluating the proposed rankings. This is because it is more meaningful
to evaluate with metrics specific to the desired features of the Test Prioritization problem.
We will further expand on metrics used for evaluation in section \ref{sec:bg-metrics-tsp}.

A common metric used to evaluate ranking algorithms is the Discounted Cumulative Gain (DCG).
\texttt{DCG@k} is defined for $k$ the truncation level as:
\begin{align*}
DCG@k = \sum_{i=1}^{k}\frac{2^{l_i}-1}{\log(i+1)}
\end{align*}
Where $l_i$ is the relevance given to the $i$-th result. As we see, this metric increases
when the first values are given a high relevance as expected. In contrast, high priority
values encountered later are penalized by $\log(i+1)$.

The truncation level just limits the considered documents for this metric.

\texttt{NDCG@k} is the Normalized version of \texttt{DCG@k}, to do so it compares against
the ideal ranking for that query and computes its corresponding \texttt{DCG@k}, let us
call it $IDCG@k$:

\begin{align*}
NDCG@k = \frac{DCG@k}{IDCG@k}
\end{align*}

The Mean Average Precision (MAP) is based on binary classification metrics. Traditionally
precision and recall are widely used for binary classification. In the context of
Information Retrieval, \emph{precision} refers to how many documents marked as relevant
are relevant by our prediction, \emph{recall} refers to how many of the relevant documents
were retrieved from all the relevant documents in the query. 

The average precision ($\text{AveP}$) represents the area under the curve of a precision-recall
plot when considering the first ranked elements:
\begin{align*}
\text{AveP} = \sum_{k=1}^{n} P(k)|R(k)-R(k-1)|
\end{align*}

Where $P(k), R(k)$ are the precision and recall obtained for the first $k$ results.

While our proposed prioritizations to label the dataset are not binary (see section \ref{s:method-prioritizing-testruns}),
they can be considered binary by giving a cutoff point for the assigned relevance.

The Expected Reciprocal Rank (\texttt{ERR@k}) metric, was proposed in \cite{10.1145/1645953.1646033}
by Chapelle, et. al..It is designed to take into consideration the relative ordering between ranked results.
In contrast to \texttt{DCG}, it does not give the same gain and discount to a fixed position.
It is defined by:

\begin{align*}
\sum_{r=1}^k \frac{1}{r} R_r \prod_{i=1}^{r-1}(1-R_i)
\end{align*}

Where, for $l_m$ the maximum priority value of the ranking:
\begin{align*}
R_i = \frac{2^{l_i}-1}{2^{l_m}}
\end{align*}

While this is the least intuitive of the metrics, it correlates better with search engine applications.
It is based on modeling the probability of a user finding its query at a given document position.

As part of our experiments described in chapter \ref{s:method}, we vary the metrics
used to train the ranking algorithms. 

We will now briefly describe the different ranking algorithms explored.

\subsubsection{Coordinate Ascent}
Coordinate Ascent is a general \textit{optimization} method, it is based on iterations defined
by maximizing the given function $f$ when fixing all coordinates but one. Formally, the $k$-th iteration,
has as $i$-th component:
\begin{align*}
x^{k+1}_i = \arg \max_{t\in\mathbb{R}} f(x^{k}_1, ..., x^{k}_{i-1}, t, x^{k}_{i+1}, ..., x^k_n)
\end{align*}

This method has similar convergence conditions as gradient descent. It was first proposed as a
ranking method by Metzler and Croft in \cite{Metzler2006LinearFM}, and it has successfully been applied
to different ranking problems. We give a short overview of how this optimization method is used as
a ranking method, but we refer to the above-mentioned article for a detailed explanation.

Among other things that differ from the traditional optimization, in \cite{Metzler2006LinearFM} they propose 
other constraints to the scoring functions and transformations to reduce the parameter space 
being optimized to find values on a simplex.

To make it more concrete, the ranking is induced by a scoring function $S$, for a document $D$ and query $Q$
 of the following form:

\begin{align*}
S(D; Q) = \Omega \cdot f(D, Q) + Z
\end{align*}

For free parameters to optimize $\Omega$, the feature vector $f(D, Q)$ and a constant $Z$. The free parameters are positive 
values such that they sum up to 1. In the ranking library used for the implementation in this project, Ranklib, $Z$ is set to zero.

Notice that this scoring function is not the function being optimized, but the ranking evaluation metric.

\subsubsection{MART}
Regression algorithms using gradient boosting were first proposed by Friedman in \cite{Friedman2001GreedyFA},
Multiple Additive Regression Trees (MART) is a gradient boosting technique further developed by Friedman and
Meulman in \cite{Friedman2003MultipleAR}.

Again, we give a very shallow explanation, as a general introduction. We refer the reader to the articles
cited above and Machine Learning literature for a more complete explanation.

In general, this is a regression technique that approximates a function by minimizing some related loss 
function. The idea is to use a linear combination of $M$ different models $h_i$ (called weak learners):

\begin{align*}
F(x) = \sum_{i}^M \gamma_i h_i(x)
\end{align*}

The idea is to fit a regression tree to approximate the target function and use the next regression tree to
approximate the residuals of the first. Afterwards, greedily compute a scale for the weak learner that minimizes
the loss function.

These residuals approximate the gradient of the loss function, effectively making this method follow the same
rationale as gradient descent to minimize the loss function.

For \emph{pointwise} ranking algorithms, regression algorithms such as MART can be used to rank
by regressing a relevance function for each of the documents to rank, minimizing some of the 
different training metrics presented in section \ref{s:bg-rnk-metrics}.

\subsubsection{LambdaMART}
LambdaMART takes its name from its constituents: LambdaRank and MART. In the previous section, we 
explained how MART works for general regression and ranking.

In contrast to \emph{pointwise} algorithms that used a relevance function to produce a ranking, 
\emph{pairwise} algorithms like LambdaMART aim to produce a comparison function between documents.

For the case of this family of algorithms, the aim is to obtain a function that given two documents $x_i$, $x_j$, 
obtains the probability that for a given query $q$, $x_i$ is ranked with higher than $x_j$: $P_{ij}$. With 
such a comparison function, sorting of the complete set of features can be performed.

To do so, the trained model is a function $f$ that only takes as input a feature $x_i$, 
and outputs a real value $f(x_i)$. To obtain the probability of the pairwise comparison a logistic function is used:

\begin{align*}
P_{ij} = \frac{1}{1+e^{-\sigma(f(x_i)-f(x_j))}}
\end{align*}

The loss function used is the cross-entropy measure. Minimizing through gradient descent is the idea behind predecessors
of this algorithm like RankNet and LambdaRank.

In LambdaMART this gradient is not computed but predicted by boosted regression trees. We refer
to \cite{lambdamart} for a more thorough explanation of how these gradients are reduced to scalars
subject to learning models.

The method has been successfully applied in diverse ranking applications, and in particular, it performed
the best in the analysis given by Bertolino, et. al. in \cite{Bertolino2020LearningtoRankVR} on Test Prioritization.

\subsubsection{RankBoost}\label{s:bg-tsp-rankboost}

First proposed by Freund, Yoav, et. al. in \cite{10.5555/945365.964285}. As MART, it uses \emph{boosting}, i. e. combining
several \emph{weak} learners into a single model.

Each of these learners predicts a relevance function, and therefore a ranking. For training, a distribution $D$ over
$X\times X$ is required. Where $X$ is the documents on a query. For this reason, this method is $O(|X|^2)$ in memory, which
can restrict the applicability of this algorithm. 

Each learner updates the distribution $D$, emphasizing the pairs that are more relevant for the algorithm
to properly order. The final relevance function then becomes a linear combination of each of these learners.
We refer to the cited article for more details and further reading.

\subsubsection{AdaRank}

First proposed in \cite{xuliadarank} by Xu and Li, it was designed to directly minimize Information Retrieval (IR) performance measures
like \texttt{MAP} and \texttt{NDCG}. It is based on AdaBoost, a binary classifier also based on \emph{boosting},
obtaining a model from the linear combination of several \emph{weak} learners.

On each iteration, it maintains a distribution of weights assigned to each of the training queries.
Such distribution is updated, increasing the values of the queries that the weak learner ranks the worst.
In this way, subsequent learners can focus on those queries for the next rounds.

The weak learners they propose are linear combinations of the metrics to minimize and the weight distribution.

\subsection{Metrics for evaluating Test Selection and Prioritization}
\label{sec:bg-metrics-tsp}

While the problem of ranking has been widely studied and metrics have been proposed for evaluating it. It is more
meaningful for our purposes to evaluate the resulting prioritization with metrics in the context of regression testing.

In this section we expand upon some of the metrics previously proposed in the literature, to evaluate the problems of Test Selection
and Prioritization.

\subsubsection{Test selection execution time}
For evaluating selection algorithms, a natural approach is to measure the time it takes to run the subset. To make 
this metric test suite independent a ratio is used:
\begin{align*}
    t_x = \frac{t_S}{t_C}
\end{align*}

Where $t_S$ is the time taken to execute the selection and $t_C$ is the time taken to execute the complete test suite.

\subsubsection{Inclusiveness}
In test selection, we do not only rely on the execution time for evaluation. Consider an arbitrarily small subset selected, it would yield good results,
but potentially it could also miss some fault-revealing test cases.

To consider this, inclusiveness is introduced:
\begin{align*}
i = \frac{|S_F|}{|T_F|}
\end{align*}

Where $S_F$ is the set of fault-revealing test cases from a selection, and $T_F$ is the set of test faults in the complete
test suite. For completeness, we define $i=1$ when there are no test faults in the given change.

A \emph{safe} test selection algorithm \cite{366926} always has $i = 1$. As every fault-revealing test is included in the selection.

\subsubsection{Selection size}
On the other hand, high inclusiveness could also be a sign of over-selecting test cases. For example, selecting the whole test
suite trivially has $i=1$. To have a measure how big the selection is, \emph{selection size} was proposed:

\begin{align*}
    ss = \frac{|S|}{|T|}
\end{align*}

A good selection algorithm strives for having a small selection size, while high inclusiveness.

\subsubsection{Time to the first failure}
Likewise, another time-related metric of interest for prioritization is the time it takes to reach the first failure:
\begin{align*}
    t_{ff} = \frac{t_F}{t_C}
\end{align*}
Where $t_F$ is the time taken to reach the first failure for the proposed prioritization.

\subsubsection{Normalized Average of the Percentage of Faults Detected}
For prioritization, only focusing on time to get to the first failure is skewed. As one could have detected the first failure
soon while prioritizing the rest of the failing test cases with low priority.

To overcome this, a widely used metric, first proposed by Elbaum, Malishevsky, and Rothermel in \cite{elbaum2002} is the 
Average of the Percentage of Faults Detected (APFD).

Normalizing this metric to the number of failures detected allows considering cases where no
selected test case was failing. This is useful for evaluating prioritizations that had previously some selection criteria applied.

It is defined by:
\begin{align*}
NAPFD = p - \frac{\sum_{i=1}^mTF_i}{nm} + \frac{p}{2n}
\end{align*}

For $p$ the ratio of detected faults on a selection against the total amount of faults,
$n$ the number of test cases, $m$ the total number of faults, and $TF_i$ the ranking of
the test case that revealed the $i$-th failure.

It represents the proportion of the test failures detected against each executed test.
We aim for this value to be close to 1, representing that the accumulated amount of 
failures detected is obtained early with the prioritization.
%
\chapter{Method}\label{s:method}

\section{Outline}

In section \ref{s:method-collecting-dataset}, we describe how we obtained the information
from the build system of \emph{Business Central}, along with the required changes to collect 
coverage information.

In section \ref{s:method-characterizing-testruns}, we describe how we obtain features for
the collected changes and test runs. These features are used to numerically represent a 
test run in a CI job, for them to be the input of the ranking algorithms.

Afterwards, in section \ref{s:method-prioritizing-testruns}, we describe how we label
the training dataset. That is, the different criteria used to assign priorities to
the given tests.

Finally, in section \ref{s:method-training-models}, we describe the training of the 
different ranking models, with different datasets.
\section{Collecting the dataset of CI executions}\label{s:method-collecting-dataset}

As initial step of our project, we collect information about the CI executions
from data stored by the DME build system and history of the repository.

For each CI job, the information extracted was:
\begin{itemize}
    \item Execution model with the tasks that the DME system used as input.
    \item Job execution properties: duration, result, and date.
    \item For each of the tasks executed by the job, information on properties: duration, result, and date.
    \item Other metainformation to identify the job in the VCS.
    \item For each of the \emph{application test} tasks, the result of each procedure run for each of the test units considered, along with information on duration of it's execution.
    \item Comparison with last merge from the target branch: path and directories where changes were made, type of changes performed, and the content of modified files.
\end{itemize}

The aim was to collect enough properties to represent the changes a developer made
with respect to different properties of the codebase for each test. Along with data to evaluate 
the prioritized tests.

\subsection{Coverage information for test runners}\label{s:method-collecting-coverage}

The information listed in the previous section was collected from real operation data
of the pipeline. As explained in section \ref{sec:app-tests-al}, there are two different
implementations of test runners that the tasks may use.

As explained in section \ref{s:bg-bc-coverage}, in the current pipeline, tests ran with the \emph{CAL test runner}
do have coverage information. However that is not the case for tests that use the \emph{AL test runner}.

A full coverage report for all the application tests was not initially available. However, as 
part of this project, modifications to the \emph{AL test runner} were done to allow 
for collecting the same kind of information\footnote{The changes done were based on 
previous work by Nikola Kukrika (nikolak@microsoft.com)}. However, these changes were not 
integrated to the pipeline. Instead the changes made to the test runner were run against snapshots of 
the codebase in a given time.

It has been discussed previously in research how coverage information may be
outdated and hard to maintain \ref{Bertolino2020LearningtoRankVR}. This is 
partly true in our case as well, however, we acknowledge that the information 
given by coverage can be valuable for our problem.

In \ref{Busjaeger2016LearningFT} a more robust approach to use coverage information is proposed, by defining
a coverage score. A feature like coverage score, mitigates for the lack of accuracy 
of the coverage information. As an additional mitigation to this problem we introduce
windows to compute such coverage scores as it will be shown in \ref{s:method-characterizing-testruns}.

\subsection{The collection process}

As a general overview, over roughly a week period, real CI jobs in this pipeline were collected. And 
sporadically between these jobs, custom jobs were executed with the required changes
to the \emph{AL test runner} to collect coverage information. 

For the results presented in this work, two of these coverage collecting jobs were 
executed and retrieved, and 172 CI jobs were collected.

Whenever coverage information is required to compute the features of a given CI job,
the coverage information used will be the closest earlier collected one.

The scale of the collected information limited the amount of jobs we were able to
collect.\footnote{The information on CI job executions amounted to 6GB of data and 
the coverage data to 44GB. }
\section{Representing tests and codebase changes by a vector}\label{s:method-characterizing-testruns}

For each of the collected CI jobs, we obtain features to characterize
the changes made by the developer, for each of the executed tests in the job.

In this section, we explain the different kinds of information obtained to
characterize them.

\subsection{File changes}

We use the following quantities to represent the changes done for each object:

\begin{itemize}
    \item Amount of new AL tables.
    \item Amount of new AL objects.
    \item Amount of modified AL objects.
    \item Amount of removed AL objects.
    \item Amount of changed tests.
    \item Amount of non-AL file changes.
    \item Amount of added lines to AL objects.
    \item Amount of removed lines from AL objects.
\end{itemize}

\subsection{Test history properties}

For each of the tests in a job we add the following historic properties:

\begin{itemize}
    \item Proportion of times the test has failed within the available data.
    \item From the past $k$ job executions, the proportion of times the test has failed.
    \item Average duration of this test within the available data.
\end{itemize}

\subsection{Coverage properties}

Using the most recent coverage information collected previous to
the given job, we compute:
\begin{itemize}
    \item Ratio between lines covered by that test and the average.
    \item The number of files changed that were covered by that test.
    \item The number of lines changed that were covered by that test within different \emph{windows}.
\end{itemize}

The lines changed were not matched exactly but by \emph{windows}. If a change on a
line nearby was covered, it was counted for such properties. We added features for
different window sizes.

The aim of such is to reduce the impact of having outdated coverage information.

In section \ref{s:future-evalp} we discuss some of the different features that 
future work could consider found in the literature.
\section{Prioritizing Test Executions in a CI job}\label{s:method-prioritizing-testruns}

As part of our training dataset for the ranking algorithms, we need to associate a priority value
to each of the tests executed in a job. This relevance induces the ranking that we desire the algorithm to 
learn. This is the process commonly referred to as \emph{dataset labeling} in the context of ML.

It is worth emphasizing that this prioritization is not the one against which the results
will be evaluated, as this would be biased. The evaluation will be given only by the 
TSP metrics presented in section \ref{sec:bg-metrics-tsp}.

A ranking can be defined by a priority function (also called relevance function): when each 
test case is assigned a priority, this value can be used to produce a ranking where such 
priority values are in increasing order.

We explain the two different approaches taken to assign priority functions. We created training datasets with both of these priority functions.

\subsection{Failure and Duration decreasing exponential priority}\label{s:method-prio-exp}

In \cite{Bertolino2020LearningtoRankVR} Bertolino, et. al. propose a priority function for each test case, based on its duration and outcome.
They define a score for the $i$-th test case $R_i$ by:
\begin{align*}
R_i = F_i + e^{-T_i}
\end{align*}

Where $F_i = 1$ if the test fails, $0$ otherwise, and $T_i$ the duration of its execution.

By design, this score ranks first failing tests and breaks ties via their execution duration.
Furthermore, by this being an exponentially decreasing function, changes
in duration have a larger effect for tests with small duration, than for test executions with
large duration.

\subsection{Coverage discrete priorities}\label{s:method-prioritizingtestcases}
We propose a discrete priority function using coverage information, since as stated in \textbf{RQ3}
we aim to understand the effect of using coverage information in different criteria of the
\emph{Learning to Rank} techniques.

Given the $i$-th test, we define $L_i$ as the number of lines covered by this test.

For a given job executing a subset of tests $\tau$, we define $\mu_{L,\tau}$ to be the mean of
all values $\{L_i\}_{i \in \tau}$. For such job, our proposed priority function prioritizes tests
in the following order:

\begin{itemize}
    \item Failing Tests
    \item New or modified tests in the job
    \item Tests that have no coverage information
    \item Tests covering changed lines, where $L_i \ge \mu_{L,\tau}$
    \item Tests covering changed lines, where $L_i < \mu_{L,\tau}$
    \item Tests where $L_i \ge \mu_{L,\tau}$
    \item Tests where $L_i < \mu_{L,\tau}$
\end{itemize}

This prioritization ranks first failing tests, and then as a conservative 
approach the modified tests in the change, and tests for which we do not have any coverage 
information, for example, new tests. Afterwards, we assign a higher priority to tests that
 traverse the changed lines on their execution, and finally the other unrelated tests.
 We break ties between them by the density of lines covered by each test since intuitively,
 a test covering more lines has a higher likelihood of failing.

As described in section \ref{s:method-collecting-dataset}, for each of these jobs we do not 
have the exact coverage information, but the most recent, previously collected. 

\section{Training ranking models}\label{s:method-training-models}

For each training performed, we varied the training metrics presented in section \ref{s:bg-rnk-metrics}. 

Additionally, for algorithms based on regression trees like MART and LambdaMART, we varied
the number of trees for gradient boosting.

As explained in previous sections, we produced different datasets changing the prioritization
criteria and the features that describe each change. As means of identification for evaluation, 
we name them as described in table \ref{f:table-naming-datasets}.

\begin{table}[h!]
    \centering
    {\renewcommand{\arraystretch}{2.5}
    \begin{tabular}{|c|c|c|}
        \hline
        \textbf{Dataset name} & \textbf{Prioritization} & \textbf{Features of each test execution} \\
        \hline
        \parbox{0.12\textwidth}{\texttt{EP-NCI}} & \parbox{0.30\textwidth}{Decreasing exponential as in \cite{Bertolino2020LearningtoRankVR}} & \parbox{0.40\textwidth}{
            \begin{itemize}
                \item Properties related to AL changes
                \item Test history properties
            \end{itemize}
        } \\
        \hline
        \parbox{0.12\textwidth}{\texttt{EP-CI}} & \parbox{0.30\textwidth}{Decreasing exponential as in \cite{Bertolino2020LearningtoRankVR}} & \parbox{0.40\textwidth}{
            \begin{itemize}
                \item Properties related to AL changes
                \item Test history properties
                \item Coverage properties
            \end{itemize}
        } \\
        \hline
        \parbox{0.12\textwidth}{\texttt{CP-NCI}} & \parbox{0.30\textwidth}{Coverage based as described in section \ref{s:method-prioritizingtestcases}} & \parbox{0.40\textwidth}{
            \begin{itemize}
                \item Properties related to AL changes
                \item Test history properties
            \end{itemize}
        } \\
        \hline
        \parbox{0.12\textwidth}{\texttt{CP-CI}} & \parbox{0.30\textwidth}{Coverage based as described in section \ref{s:method-prioritizingtestcases}} & \parbox{0.40\textwidth}{
            \begin{itemize}
                \item Properties related to AL changes
                \item Test history properties
                \item Coverage properties
            \end{itemize}
        } \\
        \hline
    \end{tabular} }
    \caption{Naming of the different datasets created.}
    \label{f:table-naming-datasets}
\end{table}

We considered these different combinations of prioritization criteria and features, in accordance with
the research question \textbf{RQ3}, as we aim to understand the effect that coverage information has 
when applying these techniques.
%
\chapter{Results and Evaluation}\label{s:results}

\section{Experiment setup}

The collection process was executed over a week. We retrieved information from
172 real CI jobs from the pipeline under study. We submitted two coverage-collecting jobs 
to the build system and collected the coverage traces.

The scale of the collected information limited the number of jobs we were able to
collect and process.\footnote{The information on CI job executions amounted to 6GB of data and 
the coverage data to 44GB. }

Each of the datasets considered: \texttt{EP-NCI}, \texttt{EP-CI}, \texttt{CP-NCI}, and \texttt{CP-CI} 
consisted of 172 queries corresponding to the collected CI jobs, from which 20\% 
of the failing test jobs were used as the validation dataset. Each of these jobs
contains around 18000 tests.

It is worth comparing the size of this dataset with the provided datasets from the literature.
Spieker et. al. in \cite{DBLP:journals/corr/abs-1811-04122} use the \emph{Paint Control} dataset consisting of 180 cycles
with an average of 65 tests each.  In \cite{Bertolino2020LearningtoRankVR}, Bertolino, et. al. consider datasets
of 522 cycles with an average of 22 tests each, which they created from the Apache Commons projects 
commit history.

The training of the different \emph{Learning to Rank} algorithms was performed with the
different datasets and varying criteria. For implementation, we used Ranklib 2.17. The training
was performed with the High-Performance Cluster from the IT University of Copenhagen. The minimum resources 
required for each of the training jobs were 115GB of RAM, which resulted in a training time with an upper limit of 2 hours
for nodes with processors with over 32 cores.

Using the trained models and obtaining the evaluation metrics was lightweight to perform on
a regular computer. This negligible process would be the overhead that the CI pipeline would have to execute
to obtain the proposed selections of these approaches.

\section{Overview of the results presented}
First, we use the NAPFD metric as defined in section \ref{sec:bg-metrics-tsp} to compare the effectiveness of the 
rankings for the Test Prioritization problem.

For each ranking algorithm, we present how the different criteria used influence the
NAPFD behavior. To emphasize, the varying criteria for the experiments were:
\begin{itemize}
    \item Priority function used to label the training dataset: either the exponential prioritization proposed by Bertolino, et. al. in \cite{Bertolino2020LearningtoRankVR}, or the discrete coverage prioritization we proposed.
    \item Features characterizing each test run: we included features related to AL file changes, test execution history, and varied either considering coverage properties or not.
    \item Training metric used for training the ranking algorithms: we use the different ranking evaluation metrics explained in section \ref{sec:bg-metrics-tsp}.
    \item For MART and LambdaMART the number of regression trees used: as explained in section \ref{s:bg-tsp-ltr-algs}, these methods combine several regression trees as weak learners. We use 5, 10, 20, and 30 trees in our experiments.
\end{itemize}

For each algorithm, we present box plots of the distribution of this metric for some of these configurations. 

Additionally, for each of these ranking algorithms and configurations, we induce selection algorithms by taking the tests
for which the model predicted the highest priorities. To determine the number of tests that we take
from a prioritization we use the following criteria:

\begin{itemize}
    \item A strictly safe selection \texttt{S-SEL}: we take tests from the prioritization until for all the evaluated jobs, every test failure is included.
    \item An above 80\% average selection \texttt{80-SEL}: we take tests from the prioritization until the average of their \emph{inclusiveness} is at least 80\%.
    \item An above 50\% average selection \texttt{50-SEL}: we take tests from the prioritization until the average of their \emph{inclusiveness} is at least 50\%.
\end{itemize}

These selections are considered to support the research question \textbf{RQ2}, to obtain
concrete quantities of the number of tests that could be omitted from execution
in this pipeline.

For a complete set of values shown in the distribution plots and more evaluation metrics, see appendix \ref{sec:appendix-evaluation-results}
and the evaluation data provided in the accompanying repository.

\section{Results for Coordinate Ascent}

Using different training metrics did not have a large impact on the behavior of this algorithm. 
However, the highest average and lowest variance of the NAPFD of different datasets ranked was obtained with the 
\texttt{NDCG@30} training metric.

Additionally, results show that using coverage information as part of the features characterizing a change
results in lower values of NAPFD. However, for the coverage priority function, the NAPFD values were 
consistently higher.

The best results for this algorithm, across the different metrics, were obtained by datasets that do not
use coverage information in the features, but that use the coverage prioritization we proposed. In the worst case
for these datasets, the NAPFD of the proposed test ranking can be as low as 67\%, but as high as 99\%.

Figures \ref{fig:coordinate-ascent-02-napfd} to \ref{fig:coordinate-ascent-06-napfd} show the box plot of the distribution
of NAPFD values for the jobs in the validation dataset for the different training metrics. 

Additionally, for \texttt{NDCG@30}, we present the distribution of values 
across each different dataset of the \emph{time to the first failure} metric in figure \ref{fig:coordinate-ascent-06-tff}. 

We observe that only in the case of the dataset with coverage information and coverage prioritization, the proposed prioritization
achieves a value as high as 88\% of the total execution time with an average of 14\%. For the remaining datasets, the first failure is detected
at most in the first 2\% of the total execution time.

Regarding the induced selections, for the training metric \texttt{NDCG@30} and \texttt{CP-CI} dataset the following results were obtained:
\begin{itemize}
    \item \texttt{S-SEL}: A safe selection was achieved with a selection size of 40\%, corresponding to 10\% of the execution time.
    \item \texttt{80-SEL}: A selection with average inclusiveness over 80\% was achieved with a selection size of 40\%, corresponding to 10\% of the execution time.
    \item \texttt{50-SEL}: A selection with average inclusiveness over 50\% was achieved with a selection size of 10\%, corresponding to 3\% of the execution time.
\end{itemize}

\begin{figure}
    \centering
    \begin{minipage}{.45\textwidth}
        \centering
        \label{fig:coordinate-ascent-02-napfd}
        \includegraphics[width=0.9\textwidth]{data/evaluation/comparing-ranking-configurations/coordinateascent-02/distribution-comparison-Selection-100-NAPFD.png}
        \parbox{0.9\textwidth}{\caption{Distribution across the different datasets of NAPFD values for the Coordinate Ascent algorithm using the \texttt{DCG@10} training metric.}}
    \end{minipage}%
    \begin{minipage}{.45\textwidth}
        \centering
        \label{fig:coordinate-ascent-03-napfd}
        \includegraphics[width=0.9\textwidth]{data/evaluation/comparing-ranking-configurations/coordinateascent-03/distribution-comparison-Selection-100-NAPFD.png}
        \parbox{0.9\textwidth}{\caption{Distribution across the different datasets of NAPFD values for the Coordinate Ascent algorithm using the \texttt{MAP} training metric.}}
    \end{minipage}%
\end{figure}

\begin{figure}
    \centering
    \begin{minipage}{.45\textwidth}
        \centering
        \label{fig:coordinate-ascent-01-napfd}
        \includegraphics[width=0.9\textwidth]{data/evaluation/comparing-ranking-configurations/coordinateascent-01/distribution-comparison-Selection-100-NAPFD.png}
        \parbox{0.9\textwidth}{\caption{Distribution across the different datasets of NAPFD values for the Coordinate Ascent algorithm using the \texttt{NDCG@10} training metric.}}
    \end{minipage}%
    \begin{minipage}{.45\textwidth}
        \centering
        \label{fig:coordinate-ascent-04-napfd}
        \includegraphics[width=0.9\textwidth]{data/evaluation/comparing-ranking-configurations/coordinateascent-04/distribution-comparison-Selection-100-NAPFD.png}
        \parbox{0.9\textwidth}{\caption{Distribution across the different datasets of NAPFD values for the Coordinate Ascent algorithm using the \texttt{NDCG@20} training metric.}}
    \end{minipage}%
\end{figure}

\begin{figure}
    \centering
    \begin{minipage}{.45\textwidth}
        \centering
        \label{fig:coordinate-ascent-06-napfd}
        \includegraphics[width=0.9\textwidth]{data/evaluation/comparing-ranking-configurations/coordinateascent-06/distribution-comparison-Selection-100-NAPFD.png}
        \parbox{0.9\textwidth}{\caption{Distribution across the different datasets of NAPFD values for the Coordinate Ascent algorithm using the \texttt{NDCG@30} training metric.}}
    \end{minipage}%
    \begin{minipage}{.45\textwidth}
        \centering
        \label{fig:coordinate-ascent-06-tff}
        \includegraphics[width=0.9\textwidth]{data/evaluation/comparing-ranking-configurations/coordinateascent-06/distribution-comparison-TimeToFirstFailure.png}
        \parbox{0.9\textwidth}{\caption{Distribution across the different datasets of times to first failure for the Coordinate Ascent algorithm using the \texttt{NDCG@30} training metric.}}
    \end{minipage}%
\end{figure}

\section{Results for LambdaMART}
For LambdaMART, using the \texttt{MAP} metric, consistently ranked with a NAPFD value of 66\% for any of the different 
trained datasets, as it can be seen in figure \ref{fig:lambdamart-13-napfd}. With the \texttt{ERR@10} metric, the training did not converge, resulting in an invalid model.

Apart from these two metrics, the other training metrics behaved similarly across the different datasets. The best NAPFD
value was obtained with the \texttt{DCG@10} metric.

Regarding the number of trees used for gradient boosting, the best performing values were obtained with 20 regression trees.

For this algorithm, using the coverage prioritization proposed yielded better NAPFD values. When using the exponential prioritization,
using coverage information as features increased the NAPFD average and reduced its variance for the majority of the experiments.

In figure \ref{fig:lambdamart-10-tff} we can see the comparison of distributions of \emph{time to the first failure},
for the metric \texttt{DCG@10} and 20 regression trees. For this configuration, the \texttt{CP-NCI} dataset,
which performed better across configurations, yields an average NAPFD value of 86\%.

The resulting induced selections are:
\begin{itemize}
    \item \texttt{S-SEL}: A safe selection was achieved with a selection size of 40\%, corresponding to 9\% of the execution time.
    \item \texttt{80-SEL}: A selection with average inclusiveness over 80\% was achieved with a selection size of 40\%, corresponding to 9\% of the execution time.
    \item \texttt{50-SEL}: A selection with average inclusiveness over 50\% was achieved with a selection size of 10\%, corresponding to 2\% of the execution time.
\end{itemize}

\begin{figure}
    \centering
    \begin{minipage}{.45\textwidth}
        \centering
        \label{fig:lambdamart-13-napfd}
        \includegraphics[width=0.9\textwidth]{data/evaluation/comparing-ranking-configurations/lambdamart-13/distribution-comparison-Selection-100-NAPFD.png}
        \parbox{0.9\textwidth}{\caption{Distribution across the different datasets of NAPFD values for the LambdaMART algorithm using the \texttt{MAP} training metric and 30 trees.}}
    \end{minipage}%
    \begin{minipage}{.45\textwidth}
        \centering
        \label{fig:lambdamart-01-napfd}
        \includegraphics[width=0.9\textwidth]{data/evaluation/comparing-ranking-configurations/lambdamart-01/distribution-comparison-Selection-100-NAPFD.png}
        \parbox{0.9\textwidth}{\caption{Distribution across the different datasets of NAPFD values for the LambdaMART algorithm using the \texttt{NDCG@10} training metric and 30 trees.}}
    \end{minipage}%
\end{figure}

\begin{figure}
    \centering
    \begin{minipage}{.45\textwidth}
        \centering
        \label{fig:lambdamart-02-napfd}
        \includegraphics[width=0.9\textwidth]{data/evaluation/comparing-ranking-configurations/lambdamart-02/distribution-comparison-Selection-100-NAPFD.png}
        \parbox{0.9\textwidth}{\caption{Distribution across the different datasets of NAPFD values for the LambdaMART algorithm using the \texttt{NDCG@10} training metric and 20 trees.}}
    \end{minipage}%
    \begin{minipage}{.45\textwidth}
        \centering
        \label{fig:lambdamart-09-napfd}
        \includegraphics[width=0.9\textwidth]{data/evaluation/comparing-ranking-configurations/lambdamart-09/distribution-comparison-Selection-100-NAPFD.png}
        \parbox{0.9\textwidth}{\caption{Distribution across the different datasets of NAPFD values for the LambdaMART algorithm using the \texttt{DCG@10} training metric and 30 trees.}}
    \end{minipage}%
\end{figure}

\begin{figure}
    \centering
    \begin{minipage}{.45\textwidth}
        \centering
        \label{fig:lambdamart-10-napfd}
        \includegraphics[width=0.9\textwidth]{data/evaluation/comparing-ranking-configurations/lambdamart-10/distribution-comparison-Selection-100-NAPFD.png}
        \parbox{0.9\textwidth}{\caption{Distribution across the different datasets of NAPFD values for the LambdaMART algorithm using the \texttt{DCG@10} training metric and 20 trees.}}
    \end{minipage}%
    \begin{minipage}{.45\textwidth}
        \centering
        \label{fig:lambdamart-17-napfd}
        \includegraphics[width=0.9\textwidth]{data/evaluation/comparing-ranking-configurations/lambdamart-17/distribution-comparison-Selection-100-NAPFD.png}
        \parbox{0.9\textwidth}{\caption{Distribution across the different datasets of NAPFD values for the LambdaMART algorithm using the \texttt{NDCG@20} training metric and 30 trees.}}
    \end{minipage}%
\end{figure}

\begin{figure}
    \centering
    \begin{minipage}{.45\textwidth}
        \centering
        \label{fig:lambdamart-18-napfd}
        \includegraphics[width=0.9\textwidth]{data/evaluation/comparing-ranking-configurations/lambdamart-18/distribution-comparison-Selection-100-NAPFD.png}
        \parbox{0.9\textwidth}{\caption{Distribution across the different datasets of NAPFD values for the LambdaMART algorithm using the \texttt{NDCG@20} training metric and 20 trees.}}
    \end{minipage}%
    \begin{minipage}{.45\textwidth}
        \centering
        \label{fig:lambdamart-21-napfd}
        \includegraphics[width=0.9\textwidth]{data/evaluation/comparing-ranking-configurations/lambdamart-21/distribution-comparison-Selection-100-NAPFD.png}
        \parbox{0.9\textwidth}{\caption{Distribution across the different datasets of NAPFD values for the LambdaMART algorithm using the \texttt{NDCG@30} training metric and 30 trees.}}
    \end{minipage}%
\end{figure}

\begin{figure}
    \centering
    \begin{minipage}{.45\textwidth}
        \centering
        \label{fig:lambdamart-22-napfd}
        \includegraphics[width=0.9\textwidth]{data/evaluation/comparing-ranking-configurations/lambdamart-22/distribution-comparison-Selection-100-NAPFD.png}
        \parbox{0.9\textwidth}{\caption{Distribution across the different datasets of NAPFD values for the LambdaMART algorithm using the \texttt{NDCG@30} training metric and 20 trees.}}
    \end{minipage}%
    \begin{minipage}{.45\textwidth}
        \centering
        \label{fig:lambdamart-10-tff}
        \includegraphics[width=0.9\textwidth]{data/evaluation/comparing-ranking-configurations/lambdamart-10/distribution-comparison-TimeToFirstFailure.png}
        \parbox{0.9\textwidth}{\caption{Distribution across the different datasets of \emph{time to the first failure} values for the LambdaMART algorithm using the \texttt{DCG@10} training metric and 20 trees.}}
    \end{minipage}%
\end{figure}

\section{Results for MART}
The results show no significant impact on the training metric used for this algorithm. The best configuration uses 30 regression trees.

As with the other algorithms, the dataset that provided the best results is the one using coverage information for prioritizing, but
not using coverage information on the feature vectors.

For one of the best performing configurations, we can see in figure \ref{fig:mart-09-tff} a distribution of
the \emph{time to the first failure} values. The best configurations achieved an average NAPFD value of 67\%.

For such configuration, the resulting induced selection has similar values as the other algorithms:
The resulting induced selections are:
\begin{itemize}
    \item \texttt{S-SEL}: A safe selection was achieved with a selection size of 40\%, corresponding to 9\% of the execution time.
    \item \texttt{80-SEL}: A selection with average inclusiveness over 80\% was achieved with a selection size of 40\%, corresponding to 9\% of the execution time.
    \item \texttt{50-SEL}: A selection with average inclusiveness over 50\% was achieved with a selection size of 10\%, corresponding to 2\% of the execution time.
\end{itemize}

\begin{figure}
    \centering
    \begin{minipage}{.45\textwidth}
        \centering
        \label{fig:mart-09-napfd}
        \includegraphics[width=0.9\textwidth]{data/evaluation/comparing-ranking-configurations/mart-09/distribution-comparison-Selection-100-NAPFD.png}
        \parbox{0.9\textwidth}{\caption{Distribution across the different datasets of NAPFD values for the MART algorithm using the \texttt{DCG@10} training metric and 30 trees.}}
    \end{minipage}%
    \begin{minipage}{.45\textwidth}
        \centering
        \label{fig:mart-13-napfd}
        \includegraphics[width=0.9\textwidth]{data/evaluation/comparing-ranking-configurations/mart-13/distribution-comparison-Selection-100-NAPFD.png}
        \parbox{0.9\textwidth}{\caption{Distribution across the different datasets of NAPFD values for the MART algorithm using the \texttt{MAP} training metric and 30 trees.}}
    \end{minipage}%
\end{figure}
\begin{figure}
    \centering
    \begin{minipage}{.45\textwidth}
        \centering
        \label{fig:mart-09-tff}
        \includegraphics[width=0.9\textwidth]{data/evaluation/comparing-ranking-configurations/mart-09/distribution-comparison-TimeToFirstFailure.png}
        \parbox{0.9\textwidth}{\caption{Distribution across the different datasets of \emph{time to the first failure} values for the MART algorithm using the \texttt{DCG@10} training metric and 30 trees.}}
    \end{minipage}%
\end{figure}

\section{Other algorithms considered}

We performed training of AdaRank, with 500 rounds. As with other approaches, we considered different training metrics. 
However, none of the resulting models converged for our dataset.

Training of RankBoost was also attempted. However, as explained in section \ref{s:bg-tsp-rankboost},
it requires at each stage to keep a distribution $D$ of memory complexity $O(n^2)$ for $n$ the number of tests
on each CI job. With the dataset used, such memory complexity exceeded the memory capacity of even the HPC nodes 
where the training was performed.%
\chapter{Conclusion}

\section{Contributions}

\subsection{Evaluation of the ranking algorithms.}

Results for the best configurations of each algorithm induce
promising selection sizes. The algorithm with most consistent behavior
was Coordinate Ascent optimizing the \texttt{NDCG@30} metric.

Inducing a safe selection on our dataset yielded a selection size of 40\%.
This implies that the amount of tests executed could be reduced by more than half.
The corresponding execution time for such selection was in average 10\% the total
execution time from executing all the tests.

While results were promising, it is relevant to note the threats to validity 
in this project described in section \ref{s:conclusion-threats}.

\subsection{Using coverage information in \emph{Learning to Rank} approaches to the Test Prioritization problem}

In literature, coverage information in \emph{Learning to Rank} approaches to 
test prioritization is not often used. With the distinction of Busjaeger in \cite{Busjaeger2016LearningFT},
studying an industrial scenario. There is no available dataset that includes such information.

The goal of comparing different configurations varying
the information provided was to understand the effect of using coverage information.

The best performing dataset across algorithms and configurations 
consistently was the one that used the proposed discrete relevance  
function given in \ref{s:method-prioritizingtestcases} that uses coverage information. 
But that does not use coverage information in the features of each test case.

For this pipeline, this seems to imply that coverage information is 
useful to determine the prioritization, but the dimension complexity 
of adding more properties related to coverage is not.

\subsection{Infrastructure to produce \emph{Learning to Rank} datasets for the \emph{Business Central} pipeline.}

The infrastructure to collect CI history information and related coverage information,
as well as the extraction mechanism of features can be found in the accompanying 
repository of this thesis project.

Increasing the strength and confidence on the evaluation can be achieved by
enhancing the dataset. This also allows for a more data-driven approach to
do analysis on the performance on the CI pipeline.%
\section{Threats to validity}\label{s:conclusion-threats}

Although the results are promising, it is important to consider
the threats to the validity of the evaluation. The most relevant threat 
is a biased dataset.

As explained in section \ref{s:method-collecting-dataset}, the dataset was 
collected over a week. Also, as previously explained, the size of 
the collected dataset and test suite restricted the training of the ranking
algorithms.

Among some of the reasons this could have biased the evaluation, is that
a week is a short time to be representative of the overall test execution dynamics.

For example, a developer working on a feature may re-run the pipeline and continuously
fail the same set of related tests. Given that history features are used, the learning algorithms
could use this as an indicator of a high priority test. Which may not be true after the 
work is done. In an online implementation of these techniques, this problem
could be mitigated by periodic retraining of the model.
%
\section{Future work}

\subsection{Usage of a prioritization technique in context of the existing sytem.}

This work focused on an offline evaluation of the techniques, for an online implementation
other technical challenges and practical considerations remain.

So far in our discussion, we did not tie how proposing a prioritization
with these techniques relates to how they are executed by the DME system explained
in section \ref{s:bc-ci-dme}.

In this section we outline the required changes, and possible strategies to use
such rankings.

Recall that the DME system, executes tasks from a given job, by
traversing each of the required dependencies defined by the model of the job. 
A single task may execute multiple test codeunits or test solutions. 

Given a ranking proposed by these techniques, we can use the induced selection
algorithm to reduce the amount of test codeunits that each of these tasks execute.

Furthermore, if the selection results on \emph{job tasks} with no test codeunits to
execute, we can remove this task, along with the dependencies that were 
only required by this task. By doing so recursively we can remove entire paths of the
job's execution. In general terms, this corresponds to the reachable vertices
of these removed vertices that are not reachable by any other non removed vertex 
on the opposite graph.

For clarity, see figure \ref{f:conc-fut-dag-removingtask}, where a DAG representing the dependencies of a job 
execution is given. Filled with black are the test tasks that after the selection
algorithm, had no tests to execute. Filled with gray are the tasks that were only
required for such tests, and therefore could be removed.

\begin{figure}
    \centering
    \def\svgwidth{0.5\columnwidth}
    \includesvg[inkscapelatex=false]{thesis/figures/network-plots/removing-tasks-model}
    \caption{Tasks and predecessors removed from the job model.}
    \label{f:conc-fut-dag-removingtask}
\end{figure}

This would be an effective use of the selection proposed, however engineering is required
to allow for such dynamic changes of a job's execution. In particular, test codeunits
are not \emph{first-class citizens} on the data model proposed by the DME, as
it is instead responsibility of the task's implementation.

For using the prioritization, other strategies can be leveraged. Since test tasks 
can be performed in parallel by the DME, a proposed prioritization of all the 
tests in the job can not be executed as given. 

Instead, the prioritization can be used locally on each task. That is, for a given
test task, and the overall prioritization, one can get a local prioritization for the
tests belonging to such task. However, the engineering to allow dynamically sorting of the test 
codeunits by the test runners is missing to implement.

Finally, the current build system allows for assigning priorities to the tasks to run,
which is taken into consideration when deciding which task to execute next from the 
set of tasks with completed dependencies. This could be dynamically assigned based on 
the prioritization of the tests being ran. Note that this would also require for the
DME system to have knowledge of the tests being run by a task.

\subsection{Tackling CI optimization from other angles}
The product studied has several CI pipelines that constitute each of the different
development cycles, from different areas that the product has. We studied a single
optimization strategy for one these CI pipelines, namely Test Selection and Prioritization.

However, different strategies exist in the literature, for instance usage of
test suite minimization to remove superfluous tests from a codebase.

Another approach could be to have learning models to predict task failure
of each of the tasks a job is comprised from. A similar feature vector as
obtained for this project, representing the change could be part of the training 
data used to train a binary classifier, for which extensive research exists.

\subsection{Further evaluation of prioritization techniques}
Reducing the amount of tests considered for each job execution for our training
dataset, would allow for a larger timespan to be considered by our evaluation.

This would increase the confidence and validity on the results presented in this project.
%%
\chapter{Future Work}\label{s:future}

\section{Usage of a prioritization technique in the context of the existing system.}

This work focused on an offline evaluation of the techniques. For an online implementation,
other technical challenges and practical considerations remain.

So far in our discussion, we did not tie how proposing a prioritization
with these techniques relates to how they are executed by the DME system, which is explained
in section \ref{s:bc-ci-dme}.

In this section, we outline the required changes, and possible strategies to use
such rankings.

Recall that the DME system, executes tasks from a given job, by
traversing each of the required dependencies defined by the model of the job. 
A single task may execute multiple test codeunits or test solutions. 

Given a ranking proposed by these techniques, we can use the induced selection
algorithm to reduce the number of test codeunits that each of these tasks executes.

Furthermore, if the selection results in \emph{job tasks} with no test codeunits to
execute, we can remove this task, along with the dependencies that were 
only required by this task. By doing so recursively we can remove entire paths of the
job's execution.

For clarity, see figure \ref{f:conc-fut-dag-removingtask}, where a DAG representing the dependencies of a job's 
execution is given. Nodes filled with black are the test tasks that after the selection
algorithm, had no tests to execute. Nodes filled with gray are the tasks that were only
required for such tests, and therefore could be removed.

\begin{figure}
    \centering
    \def\svgwidth{0.5\columnwidth}
    \includesvg[inkscapelatex=false]{thesis/figures/network-plots/removing-tasks-model}
    \caption{Tasks and predecessors removed from the job model.}
    \label{f:conc-fut-dag-removingtask}
\end{figure}

This would be an effective use of the selection proposed, however, engineering is required
to allow for such dynamic changes in a job's execution. 

Since test tasks can be performed in parallel by the DME system, a 
proposed prioritization of all the tests in the job can not be executed as given. 
Instead, the prioritization can be used locally on each task. For a given
test task and the overall prioritization, one can get a local prioritization for the
tests belonging to such task. However, the engineering that allows dynamically 
sorting the test codeunits in a single task is also missing.

Finally, the current build system allows for assigning priorities to the tasks to run,
which is taken into consideration when deciding which task to execute next from the 
set of tasks with completed dependencies. This could be dynamically assigned based on 
the prioritization of the tests being run. 

The main engineering challenge to allow for the usage of these proposals is to make
tests \emph{first-class citizens} of the data model proposed by the DME system.
Currently, the DME system has no knowledge of the tests being run by a task, as
it is instead the responsibility of the task's implementation.

\section{Tackling CI optimization from other angles}
\emph{Business Central} has several CI pipelines that constitute each of the different
development cycles, from different areas that the product has. We studied a single
optimization strategy for one of these CI pipelines, namely Test Selection and Prioritization.

However, different strategies exist in the literature, for instance, the usage of
test suite minimization to remove superfluous tests from a codebase.

Another approach could be to have learning models to predict the task failure
of each of the tasks a job is comprised from. A similar feature vector as
obtained for this project representing the change could be part of the training 
data used to train binary classifiers.

The approach taken on this work, to obtain a meaningful representation of the
changes done to a codebase can be further leveraged for future analysis of
this CI pipeline.

\section{Further evaluation of test prioritization techniques}\label{s:future-evalp}
Reducing the number of tests considered for each job execution for our training
dataset would allow for a larger timespan to be considered in our evaluation.
As each job can become more manageable for the learning algorithms and feature
extraction infrastructure. 

For instance, in this project, we were not able
to perform training of RankBoost, as its memory complexity is squared in the
number of tests.

This would also increase the confidence and validity of the results presented in this 
project. Having a larger dataset allows for multiple validation datasets, to increase the
confidence of the proposed selection sizes in this work.

Other approaches that avoid retraining like Reinforcement Learning could be 
leveraged and be the subject of future work with the produced dataset, although
comparison in existing literature favors \emph{Learning to Rank} approaches \cite{Bertolino2020LearningtoRankVR}.

In regards to \emph{Learning to Rank} approaches, more properties can be
obtained to give a vector representation of a codebase change and tests. 
For instance, in \cite{Bertolino2020LearningtoRankVR}, Bertolino, et. al. propose more 
code metrics that were not in scope for this project. Examples include 
cyclomatic complexity, number of public or private methods, number of 
new functions, and others. Other interesting features were explored by Busjaeger et. al. in \cite{Busjaeger2016LearningFT}, 
where besides coverage score, they use \emph{text similarity} metrics between the 
names of the test procedures and paths of the files or their content. 

Adding more useful features to this dataset would allow for more
robust criteria for the learning algorithms.%
\bibliographystyle{plain}
\bibliography{thesis/references}
\appendix

\chapter{The AL Language}
\label{sec:appendix-allang}

One of the main design aspects of \emph{Business Central} (BC) is that all business logic is written in 
a custom Application Language (AL), hiding all implementation details dealing with the technology.

Extensions can be provided for the \emph{Business Central} runtime to add business functionality
and to fit customers' requirements as much as desired. Such extensions are not only written by
Microsoft, as users can customize their experience via extensions as required.

This effectively creates an ecosystem of AL software, for diverse use cases of BC as an ERP.

Adding a language abstraction forces abstraction layers to what is possible within the language
and effectively separates business logic from technology implementation specifics.
This has allowed the product to evolve throughout the years and experience several changes 
that with a different design would be harder to achieve, for example: changing the 
database engine, or migrating to a cloud environment. Additionally, the language continues
to evolve and be actively maintained by the team, to keep up with the requirements
a modern language infrastructure requires.

Interestingly, this approach has been successfully used by other products as well. For example,
another ERP: SAP, has the language ABAP. A custom Domain Specific Language (DSL) resembling COBOL
to extend the functionality of the system.

In this section, we give a brief tour of AL, \emph{Business Central}'s custom DSL for modifying
its runtime. For a complete reference, we refer to Microsoft's documentation \cite{bcaldocs}.

\section{How does AL look?}

In terms of syntax, AL resembles Pascal. It is an imperative, and procedural language. 
As a quick overview, we show some of the basic building blocks of the language
in the following figures.

As you see, variables are \emph{typed}. The \texttt{Record} variable type corresponds to
records in a table on the underlying database of the system. As we will explain later,
a user can also define the tables to use in this language.

\begin{figure}
    \begin{minipage}{.45\textwidth}
        \begin{Verbatim}[fontsize=\tiny]
var
    myInt: Integer;
    isValid: Boolean;
        \end{Verbatim}
        \parbox{0.9\textwidth}{
            \caption{Variable declaration in AL.}
        \label{fig:al-var-decl}
            }
    \end{minipage}%
    \begin{minipage}{.45\textwidth}
        \begin{Verbatim}[fontsize=\tiny]
Amount := Total * myInt;
        \end{Verbatim}
        \caption{Assignment and operations in AL.}
    \end{minipage}
\end{figure}

\begin{figure}
    \begin{minipage}{.45\textwidth}
        \begin{Verbatim}[fontsize=\tiny]
if x = y then begin  
    x := a;  
    y := b;  
end else 
    y := b;   
        \end{Verbatim}
        \parbox{0.9\textwidth}{\caption{Branching in AL.}}
    \end{minipage}%
    \begin{minipage}{.45\textwidth}
        \begin{Verbatim}[fontsize=\tiny]
procedure MyProcedure(Arg1: Integer; 
           Arg2: Boolean): Integer
begin
    if Arg2 then
        exit(-Arg1)
    exit(Arg1)
end
        \end{Verbatim}
        \parbox{0.9\textwidth}{
            \caption{Declaring a procedure in AL.}
        \label{fig:al-proc-decl}
            }
    \end{minipage}%

\end{figure}

\section{AL objects and BC's runtime}
Objects in AL are not the general objects as understood in the Object-Oriented Programming paradigm. Instead,
they correspond to different units of BC's functionality.

Every code element in AL belongs to some object. To see the syntax of how to declare them, see figure \ref{f:appendix-alobject}.
It requires an \texttt{ObjectID} a positive integer, an \texttt{ObjectName} a string identifier, and an \texttt{ObjectType}.

\begin{figure}
    \begin{Verbatim}[fontsize=\tiny]
<ObjectType> <ObjectID> <ObjectName>
{
    // Definition of the object
}
    \end{Verbatim}
    \caption{Syntax to define an AL object.}
    \label{f:appendix-alobject}
\end{figure}

We will show some of these \texttt{ObjectType}s, and their effect on the runtime.

\subsection{The \texttt{Page} type}

Objects of \texttt{Page} type, correspond to interactive interfaces the user experience within the product.

In figure \ref{f:app-al-page-alcode} you can see the AL code used to define the page a user can interact with 
shown in figure \ref{f:app-al-page}

\begin{figure}
    \begin{Verbatim}[fontsize=\tiny]
page 379 "Bank Acc. Reconciliation"
{
    Caption = 'Bank Acc. Reconciliation';
    PageType = ListPlus;
    PromotedActionCategories = 'New,Process,Report,Bank,Matching,Posting';
    SaveValues = false;
    SourceTable = "Bank Acc. Reconciliation";
    SourceTableView = WHERE("Statement Type" = CONST("Bank Reconciliation"));

    layout
    {
        // ...
    }
}
    \end{Verbatim}
    \caption{Example of an AL page.}
    \label{f:app-al-page-alcode}
\end{figure}

\begin{figure}
    \centering
    \includegraphics[width=0.9\textwidth]{thesis/figures/images/example-al-page.png}
    \caption{The interface with which the user interacts, as defined by the AL page from figure \ref{f:app-al-page-alcode}.}
    \label{f:app-al-page}
\end{figure}

\subsection{The \texttt{Table} type}

Objects of \texttt{Table} type, correspond to persistent storage in the system. Defining this object corresponds
to creating a table in the underlying SQLServer database.

Having defined a table, a developer can now use \texttt{Record} type variables of such table,
to manipulate and use this table as required, effectively acting as a data layer abstraction.

In figure \ref{f:app-al-table-alcode} you can see the definition of a table, and in figure \ref{f:app-al-record-usage}
how data can be manipulated by the usage of a \texttt{Record} variable.

\begin{figure}
    \begin{Verbatim}[fontsize=\tiny]
table 273 "Bank Acc. Reconciliation"
{
    Caption = 'Bank Acc. Reconciliation';
    DataCaptionFields = "Bank Account No.", "Statement No.";
    LookupPageID = "Bank Acc. Reconciliation List";
    Permissions = TableData "Bank Account" = rm,
                  TableData "Data Exch." = rimd;

    fields
    {
        field(1; "Bank Account No."; Code[20])
        {
            // ...
        }
        // ...
    }
    // ...
}
    \end{Verbatim}
    \caption{Example of an AL Table.}
    \label{f:app-al-table-alcode}
\end{figure}

\begin{figure}
    \begin{Verbatim}[fontsize=\tiny]
    // ...
    CurrPage.Update(false);
    if not BankAccReconciliation.IsEmpty() then begin
        BankAccReconciliation.Validate("Statement Ending Balance", 0.0);
        BankAccReconciliation.Modify();
    end;
    // ...
    \end{Verbatim}
    \caption{Example of using a record \texttt{BankAccReconciliation} of the type defined by the table in figure \ref{f:app-al-table-alcode}.}
    \label{f:app-al-record-usage}
\end{figure}

\subsection{The \texttt{Codeunit} type}

Objects of \texttt{Codeunit} type, correspond to logical units of functionality, much like \emph{modules} in 
other languages. They are a set of procedures that can be called from anywhere else within the AL codebase
with certain access modifiers.

In figure \ref{f:app-al-codeunit} you can see a procedure defined in a codeunit and in figure \ref{f:app-al-codeunit-usage}
an example of it being used from a different AL object.

\begin{figure}
    \begin{Verbatim}[fontsize=\tiny]
codeunit 18 "Financial Report Mgt."
{

    TableNo = "Financial Report";

    var
        AccSchedLine: Record "Acc. Schedule Line";
        FinRepPrefixTxt: Label 'FIN.REP.', MaxLength = 10, // ...
        TwoPosTxt: Label '%1%2', Locked = true;
    // ...
    procedure XMLExchangeExport(FinancialReport: Record "Financial Report")
    var
        ConfigPackage: Record "Config. Package";
        ConfigXMLExchange: Codeunit "Config. XML Exchange";
    begin
        AddFinancialReportToConfigPackage(FinancialReport.Name, ConfigPackage);
        Commit();
        ConfigXMLExchange.ExportPackage(ConfigPackage);
    end;


    local procedure AddFinancialReportToConfigPackage(FinancialReportName: Code[10]; var ConfigPackage //...
    // ...
}
    \end{Verbatim}
    \caption{Example of an AL Codeunit.}
    \label{f:app-al-codeunit}
\end{figure}

\begin{figure}
    \begin{Verbatim}[fontsize=\tiny]
    // ...
    trigger OnAction()
    var
        FinancialReportMgt: Codeunit "Financial Report Mgt.";
    begin
        FinancialReportMgt.XMLExchangeExport(Rec);
    end;
    // ...
    \end{Verbatim}
    \caption{Usage of the procedure defined in figure \ref{f:app-al-codeunit} in a different AL object}
    \label{f:app-al-codeunit-usage}
\end{figure}

\section{AL Tests}

Of particular relevance for this project, is how tests are defined in this language. Tests in AL are defined
 on AL objects of type \texttt{Codeunit} with appropriate annotations.

Depending on the scope of a test, these tests can be integration tests or unit tests. See
an example of a test in figure \ref{f:app-al-test}.

\begin{figure}
    \begin{Verbatim}[fontsize=\tiny]
codeunit 134141 "ERM Bank Reconciliation"
{
    Permissions = TableData "Bank Account Ledger Entry" = ri,
                  TableData "Bank Account Statement" = rimd;
    Subtype = Test;
    TestPermissions = NonRestrictive;
    // ...
    [Test]
    [Scope('OnPrem')]
    procedure BankAccReconciliationBalanceToReconcile()
    var
        BankAccReconciliation: Record "Bank Acc. Reconciliation";
        GenJournalLine: Record "Gen. Journal Line";
        BankAccReconciliationPage: TestPage "Bank Acc. Reconciliation";
        BalanceToReconcile: Decimal;
        i: Integer;
    begin
        // [SCENARIO 363054] "Balance to Reconcile" does not include amounts from Posted Bank Reconciliations
        Initialize();

        // [GIVEN] Posted Bank Reconciliation A with Amount X
        CreateAndPostGenJournalLine(GenJournalLine, CreateBankAccount);
        CreateSuggestedBankReconc(BankAccReconciliation, GenJournalLine."Bal. Account No.", false);
        LibraryERM.PostBankAccReconciliation(BankAccReconciliation);

        // [GIVEN] Bank Reconciliation B with Amount Y
        for i := 1 to LibraryRandom.RandInt(5) do begin
            CreateAndPostGenJournalLine(GenJournalLine, GenJournalLine."Bal. Account No.");
            BalanceToReconcile += GenJournalLine.Amount;
        end;
        Clear(BankAccReconciliation);
        CreateSuggestedBankReconc(BankAccReconciliation, GenJournalLine."Bal. Account No.", false);

        // [WHEN] Bank Reconciliation B page is opened
        LibraryLowerPermissions.AddAccountReceivables;
        BankAccReconciliationPage.OpenView;
        BankAccReconciliationPage.GotoRecord(BankAccReconciliation);

        // [THEN] "Balance To Reconcile" = Y.
        Assert.AreEqual(
          -BalanceToReconcile,
          BankAccReconciliationPage.ApplyBankLedgerEntries.BalanceToReconcile.AsDEcimal,
          StrSubstNo(
            WrongAmountErr, BankAccReconciliationPage.ApplyBankLedgerEntries.BalanceToReconcile.Caption,
            -BalanceToReconcile));
    end;
    // ...
}
    \end{Verbatim}
    \caption{Example of a test codeunit and test procedure.}
    \label{f:app-al-test}
\end{figure}

The test infrastructure required for running these scenarios with different settings is also maintained
by the team and written in AL itself. 

We have just scratched the surface with this brief introduction, as it is meant to give a general idea of the
type of programs and tests that this thesis project centers around.


\chapter{Evaluation results}
\label{sec:appendix-evaluation-results}%

\end{document}